\chapter{太极拳}

\section{陈氏太极拳图说凡例}

\begin{enumerate}
  \item 学太极拳不可不敬。不敬则外慢师友,内慢身体。心不敛束,如何能学艺?
  \item 学太极拳不可狂。狂则生事。不但手不可狂,即言亦不可狂;外面形迹必带儒雅风气,不然,狂于外必失于中。
  \item 学太极拳不可满。满则招损。俗语云:天外还有天。能谦则虚心受教,人岂不乐告之以善哉!积众以为善,善斯大矣!
  \item 学太极拳着着当细心揣摩。一着不揣摩,则此势机致情理终于茫昧。即承上启下处,尤当留心。此处不留心,则来脉不真,转关亦不灵动,一着自为一着,不能自始至终一气贯通矣!不能一气贯通,则与太和元气终难问津!
  \item 学太极拳,先学读书。书理明白,学拳自然容易。
  \item 学太极拳,学阴阳开合而已。吾身中自有本然之阴阳开合,非教者所能加损也!复其本然,教者即止(教者教以规矩,即大中至正之理)。
  \item 太极拳虽无大用处,然当今之世,列强争雄,若无武艺,何以保存?惟取是书演而习之,于陆军步伐止齐之法,不无小补。我国苟人人演习,或遇交手仗,敌虽强盛,其奈我何?是亦保存国体之一道也!有心者,勿以刍荛之言弃之。
  \item 学太极拳不可借以为盗窃抢夺之资、奸情採花之用。如借以抢夺採花,是天夺之魄,鬼神弗佑,而况人乎?天下孰能容之?
  \item 学太极拳不可凌厉欺压人。一凌厉欺压,即犯众恶,罪之魁也。 
\end{enumerate}
