\section{天干}

\subsection{十天干}
\begin{table}[H]
  \centering
  \caption{十天干}
  \begin{tabular}{|c|c|c|c|c|c|c|c|c|c|c|}
    \hline & 甲 & 乙 & 丙 & 丁 & 戊 & 己 & 庚 & 辛 & 壬 & 癸 \\
    \hline 五行属性 & \multicolumn{2}{c|}{木} & \multicolumn{2}{c|}{火} & \multicolumn{2}{c|}{土} & \multicolumn{2}{c|}{金} & \multicolumn{2}{c|}{水} \\
    \hline 五运属性 & 土 & 金 & 水 & 木 & 火 & 土 & 金 & 水 & 木 & 火 \\
    \hline
  \end{tabular}
\end{table}

\subsection{五合}
\begin{table}[H]
  \centering
  \caption[]{五合\footnotemark}
  \begin{tabular}{|c|c|c|c|c|}
    \hline 甲 & 乙 & 丙 & 丁 & 戊 \\
    \hline 己 & 庚 & 辛 & 壬 & 癸 \\
    \hline 土 & 金 & 水 & 木 & 火 \\
    \hline
  \end{tabular}
\end{table}
\footnotetext{口诀:甲己合土乙庚金,丁壬合木水丙辛,戊癸合火细分明,天干五合推五运。}

\subsection{四冲}

\begin{table}[H]
  \centering
  \caption[]{四冲\footnotemark}
  \begin{tabular}{|c|c|c|c|}
    \hline 甲 & 乙 & 丙 & 丁 \\
    \hline 庚 & 辛 & 壬 & 癸 \\
    \hline
  \end{tabular}
\end{table}
\footnotetext{口诀:甲庚对头乙攻辛,壬丙相冲癸寻丁。}
