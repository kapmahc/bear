\section{五行}
最早见于《尚书》中的《甘誓》、《洪范》。
\begin{enumerate}
  \item 木:曲直
  \item 火:炎上
  \item 金:从革
  \item 水:润下
  \item 土:稼穑\footnote{读se}
\end{enumerate}
\subsection{阴阳五行学说}
\begin{enumerate}
  \item 水:太阴
  \item 金:少阴
  \item 火:太阳
  \item 木:少阳
  \item 土:四象生灭之地
\end{enumerate}
战国时期阴阳家合并为阴阳五行学说。
\subsection{生}
 土->金->水->木->火->土
\footnote{强助弱 旺助衰}
\subsection{克}
金->木->土->水->火->金
\footnote{强损弱 旺损衰 }
\subsection{乘}
\footnote{生过伤己}
\subsection{侮}
\footnote{反克为侮}
