\section{六十四卦}

南宋理學家朱熹著有《周易本義·卦名次序歌》:
\begin{quote}
    乾坤屯蒙需訟師,比小畜兮履泰否;\\
    同人大有謙豫隨,蠱臨觀兮噬嗑賁;\\
    剝復無妄大畜頤,大過坎離三十備。\\
    \\
    咸恆遯兮及大壯,晉與明夷家人睽;\\
    蹇解損益夬姤萃,升困井革鼎震繼;\\
    艮漸歸妹豐旅巽,兌渙節兮中孚至;\\
    小過既濟兼未濟,是為下經三十四。\\
\end{quote}

\begin{table}[H]
  \centering
  \caption{六十四卦}
  \tiny
  \begin{tabular}{|r|r|r|r|r|r|r|r|r|r|}
    \hline 坤\trigram{7}(地) & 艮\trigram{6}(山) & 坎\trigram{5}(水) & 巽\trigram{4}(風) & 震\trigram{3}(雷) & 離\trigram{2}(火) & 兌\trigram{1}(泽) & 乾\trigram{0}(天) & ←上卦↓下卦 \\
    \hline 坤為地\iching{1} & 山地剥\iching{22} & 水地比\iching{7} & 風地觀\iching{19} & 雷地豫\iching{15} & 火地晋\iching{34} & 泽地萃\iching{44} & 天地否\iching{11} & 坤\trigram{7}(地) \\
    \hline 地山謙\iching{14} & 艮為山\iching{51} & 水山蹇\iching{38} & 風山漸\iching{52} & 雷山小過\iching{61} & 火山旅\iching{55} & 泽山咸\iching{30} & 天山遯\iching{32} & 艮\trigram{6}(山) \\
    \hline 地水師\iching{6} & 山水蒙\iching{3} & 坎為水\iching{28} & 風水渙\iching{58} & 雷水解\iching{39} & 火水未濟\iching{63} & 泽水困\iching{46} & 天水訟\iching{5} & 坎\trigram{5}(水) \\
    \hline 地風升\iching{45} & 山風蠱\iching{17} & 水風井\iching{47} & 巽為風\iching{56} & 雷風恒\iching{31} & 火風鼎\iching{49} & 泽風大過\iching{27} & 天風姤\iching{43} & 巽\trigram{4}(風) \\
    \hline 地雷復\iching{23} & 山雷頤\iching{26} & 水雷屯\iching{2} & 風雷益\iching{41} & 震為雷\iching{50} & 火雷噬嗑\iching{20} & 泽雷随\iching{16} & 天雷无妄\iching{24} & 震\trigram{3}(雷) \\
    \hline 地火明夷\iching{35} & 山火賁\iching{21} & 水火既濟\iching{62} & 風火家人\iching{36} & 雷火丰\iching{54} & 離為火\iching{29} & 泽火革\iching{48} & 天火同人\iching{12} & 離\trigram{2}(火) \\
    \hline 地泽臨\iching{18} & 山泽損\iching{40} & 水泽節\iching{59} & 風泽中孚\iching{60} & 雷泽歸妹\iching{53} & 火泽睽\iching{37} & 兌為泽\iching{57} & 天泽履\iching{9} & 兌\trigram{1}(泽) \\
    \hline 地天泰\iching{10} & 山天大畜\iching{25} & 水天需\iching{4} & 風天小畜\iching{8} & 雷天大壮\iching{33} & 火天大有\iching{13} & 泽天夬\iching{42} & 乾為天\iching{0} & 乾\trigram{0}(天) \\
    \hline
  \end{tabular}
\end{table}

\subsection{错卦}
\begin{table}[H]
  \centering
  \caption[]{错卦\footnotemark}
  \begin{tabular}{|r|r|r|r|}
    \hline 乾\iching{0} & 坤\iching{1} \\
    \hline 坎\iching{28} & 离\iching{29} \\
    \hline 大过\iching{27} & 颐\iching{26} \\
    \hline 小过\iching{61} & 中孚\iching{60} \\
    \hline
  \end{tabular}
\end{table}
\footnotetext{将正卦各爻阴阳互变另成一卦}

\subsection{四正之卦}
\begin{table}[H]
  \centering
  \caption[]{四正之卦}
  \begin{tabular}{|r|r|}
    \hline 泰\iching{10} & 否\iching{11} \\
    \hline 既濟\iching{62} & 未濟\iching{63} \\
    \hline
  \end{tabular}
\end{table}

\subsection{四隅之卦}
\begin{table}[H]
  \centering
  \caption[]{四隅之卦}
  \begin{tabular}{|r|r|}
    \hline 歸妹\iching{53} & 漸\iching{52} \\
    \hline 蠱\iching{17} & 随\iching{16} \\
    \hline
  \end{tabular}
\end{table}

\subsection{综卦}
《序卦》64卦,除乾坤、坎离、大过颐、小过中孚8个卦相错之外,其余56个皆相综;
四正之卦否、泰、既济、未济,四隅之卦归妹、渐、随、蛊,比八卦可错可综,但是文王归为综卦;
故(56/2+8)/2=18,上下经皆18卦。

\subsection{八卦变六十四卦图}
\begin{table}[H]
  \centering
  \small
  \caption[]{八卦变六十四卦图}
  \begin{tabular}{|r|r|r|r|r|r|r|r|r|}
    \hline 归本卦 & 复还四爻变 & 五爻变 & 四爻变 & 三爻变 & 二爻变 & 初爻变 & \\
    \hline 火天大有 & 火地晋 & 山地剥 & 风地观 & 天地否 & 天山遁 & 天风姤  & 乾一变 \\
    \hline 雷泽归妹 & 雷山小过 & 地山谦 & 水山蹇 & 泽山咸 & 泽地萃 & 泽水困 & 兑二变 \\
    \hline & & & & & & & 离三变 \\
    \hline & & & & & & & 震四变 \\
    \hline & & & & & & & 巽五变 \\
    \hline & & & & & & & 坎六变 \\
    \hline & & & & & & & 艮七变 \\
    \hline & & & & & & & 坤八变 \\
    \hline
  \end{tabular}
\end{table}
