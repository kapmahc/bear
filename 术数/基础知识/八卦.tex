\section{八卦}

\subsection{卦名}
乾\trigram{0} 兑\trigram{1} 离\trigram{2} 震\trigram{3}
巽\trigram{4} 坎\trigram{5} 艮\trigram{6} 坤\trigram{7}

\paragraph{宋\quad 朱熹《周易本义》八卦取象歌}
\begin{quote}
        乾三连,坤六断;
        震仰盂,艮覆碗;
        離中虚,坎中满;
        兌上缺,巽下断。
\end{quote}


\subsection{伏羲八卦方位图}
\begin{pspicture}(10,10)
\pspolygon(4,2)(6,2)(8,4)(8,6)(6,8)(4,8)(2,6)(2,4)
\uput[u](5,7.9){乾\trigram{0}(1)}
\uput[u](2.5,7){兑\trigram{1}(2)}
\uput[u](1.2,4.5){离\trigram{2}(3)}
\uput[u](2.3,2.5){震\trigram{3}(4)}
\uput[u](5,1.3){坤\trigram{7}(8)}
\uput[u](7.8,2.5){艮\trigram{6}(7)}
\uput[u](8.8,4.5){坎\trigram{5}(6)}
\uput[u](7.7,7){巽\trigram{4}(5)}
\end{pspicture}

\subsection{文王八卦方位图}
\begin{pspicture}(10,10)
\pspolygon(4,2)(6,2)(8,4)(8,6)(6,8)(4,8)(2,6)(2,4)
\uput[u](5,7.9){离\trigram{2}(南)}
\uput[u](2.5,7){巽\trigram{4}}
\uput[u](1.2,4.5){震\trigram{3}(东)}
\uput[u](2.3,2.5){艮\trigram{6}}
\uput[u](5,1.3){坎\trigram{5}(北)}
\uput[u](7.8,2.5){乾\trigram{0}}
\uput[u](8.8,4.5){兑\trigram{1}(西)}
\uput[u](7.7,7){坤\trigram{7}}
\end{pspicture}

\subsection{孔子太极生两仪四象八卦图}
\begin{table}[H]
  \centering
  \caption[]{孔子太极生两仪四象八卦图}
  \begin{tabular}{|c|c|c|c|c|c|c|c|}
    \hline \multicolumn{8}{|c|}{两仪图} \\
    \hline \multicolumn{4}{|c|}{阴仪\dejavusans\char"268B} & \multicolumn{4}{c|}{阳仪\dejavusans\char"268A} \\
    \hline \multicolumn{8}{|c|}{四象图} \\
    \hline \multicolumn{2}{|c|}{太阴\dejavusans\char"268F} & \multicolumn{2}{c|}{少阳\dejavusans\char"268E} & \multicolumn{2}{c|}{少阴\dejavusans\char"268D} & \multicolumn{2}{c|}{太阳\dejavusans\char"268C} \\
    \hline \multicolumn{8}{|c|}{八卦图}\\
    \hline 坤(八)\trigram{7} & 艮(七)\trigram{6} & 坎(六)\trigram{5} & 巽(五)\trigram{4} & 震(四)\trigram{3} & 离(三)\trigram{2} & 兑(二)\trigram{1} & 乾(一)\trigram{0} \\
    \hline
  \end{tabular}
\end{table}

\subsection{文王八卦次序}
\paragraph{乾父} 震长男\footnote{得乾初爻} 坎中男\footnote{得乾中爻} 艮少男\footnote{得乾上爻}
\paragraph{坤母} 巽长女\footnote{得坤初爻} 离中女\footnote{的坤中爻} 兑少女\footnote{得坤上爻}
