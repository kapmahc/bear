\section{历法}
\subsection{甲子}
\begin{table}[H]
  \centering
  \caption[]{六十甲子\footnotemark}
  \begin{tabular}{|c|c|c|c|c|c|}
    \hline 01甲子 & 11甲戌 & 21甲申 & 31甲午 & 41甲辰 & 51甲寅 \\
    \hline 02乙丑 & 12乙亥 & 22乙酉 & 32乙未 & 42乙巳 & 52乙卯 \\
    \hline 03丙寅 & 13丙子 & 23丙戌 & 33丙申 & 43丙午 & 53丙辰 \\
    \hline 04丁卯 & 14丁丑 & 24丁亥 & 34丁酉 & 44丁未 & 54丁巳 \\
    \hline 05戊辰 & 15戊寅 & 25戊子 & 35戊戌 & 45戊申 & 55戊午 \\
    \hline 06己巳 & 16己卯 & 26己丑 & 36己亥 & 46己酉 & 56己未 \\
    \hline 07庚午 & 17庚辰 & 27庚寅 & 37庚子 & 47庚戌 & 57庚申 \\
    \hline 08辛未 & 18辛巳 & 28辛卯 & 38辛丑 & 48辛亥 & 58辛酉 \\
    \hline 09壬申 & 19壬午 & 29壬辰 & 39壬寅 & 49壬子 & 59壬戌 \\
    \hline 10癸酉 & 20癸未 & 30癸巳 & 40癸卯 & 50癸丑 & 60癸亥 \\
    \hline
  \end{tabular}
\end{table}
\footnotetext{《素问·六微旨大论篇》:天气始于甲,地气始于子,子田相合,命曰岁立,谨候其时,气可与期。}

\subsection{计年}
\paragraph{(公元年数-3) / 60 = 干}
\paragraph{(公元年数-3) \% 60 = 支}

\subsection{计月}
地支按照农历月份;
\begin{table}[H]
  \centering
  \caption[]{月干表\footnotemark}
  \begin{tabular}{|c|c|c|c|c|c|}
    \hline 年干 & 甲、已 & 乙、庚 & 丙、辛 & 丁、壬 & 戊、癸 \\
    \hline 正月 & 丙寅 & 戊寅 & 庚寅 & 壬寅 & 甲寅 \\
    \hline 二月 & 丁卯 & 已卯 & 辛卯 & 癸卯 & 乙卯 \\
    \hline 三月 & 戊辰 & 庚辰 & 壬辰 & 甲辰 & 丙辰 \\
    \hline 四月 & 已巳 & 辛巳 & 癸巳 & 乙巳 & 丁巳 \\
    \hline 五月 & 庚午 & 壬午 & 甲午 & 丙午 & 戊午 \\
    \hline 六月 & 辛未 & 癸未 & 乙未 & 丁未 & 已未 \\
    \hline 七月 & 壬申 & 甲申 & 丙申 & 戊申 & 庚申 \\
    \hline 八月 & 癸酉 & 乙酉 & 丁酉 & 已酉 & 辛酉 \\
    \hline 九月 & 甲戌 & 丙戌 & 戊戌 & 庚戌 & 壬戌 \\
    \hline 十月 & 乙亥 & 丁亥 & 已亥 & 辛亥 & 癸亥 \\
    \hline 冬月 & 丙子 & 戊子 & 庚子 & 壬子 & 甲子 \\
    \hline 腊月 & 丁丑 & 已丑 & 辛丑 & 癸丑 & 乙丑 \\
    \hline
  \end{tabular}
\end{table}
\footnotetext{“五虎建元”歌诀:甲己起丙寅,乙庚起戊寅,丙辛起庚寅,丁壬起壬寅,戊癸起甲寅。}

\subsection{计时}
地支按照时辰;
\begin{table}[H]
  \centering
  \caption[]{时干表\footnotemark}
  \begin{tabular}{|c|c|c|c|c|c|c|}
    \hline \multicolumn{2}{|r|}{日干} & 甲、已 & 乙、庚 & 丙、辛 & 丁、壬 & 戊、癸 \\
    \hline 23-01 & 子 & 甲子 & 丙子 & 戊子 & 庚子 & 壬子 \\
    \hline 01-03 & 丑 & 乙丑 & 丁丑 & 己丑 & 辛丑 & 癸丑 \\
    \hline 03-05 & 寅 & 丙寅 & 戊寅 & 庚寅 & 壬寅 & 甲寅 \\
    \hline 05-07 & 卯 & 丁卯 & 已卯 & 辛卯 & 癸卯 & 乙卯 \\
    \hline 07-09 & 辰 & 戊辰 & 庚辰 & 壬辰 & 甲辰 & 丙辰 \\
    \hline 09-11 & 巳 & 已巳 & 辛巳 & 癸巳 & 乙巳 & 丁巳 \\
    \hline 11-13 & 午 & 庚午 & 壬午 & 甲午 & 丙午 & 戊午 \\
    \hline 13-15 & 未 & 辛未 & 癸未 & 乙未 & 丁未 & 已未 \\
    \hline 15-17 & 申 & 壬申 & 甲申 & 丙申 & 戊申 & 庚申 \\
    \hline 17-19 & 酉 & 癸酉 & 乙酉 & 丁酉 & 已酉 & 辛酉 \\
    \hline 19-21 & 戌 & 甲戌 & 丙戌 & 戊戌 & 庚戌 & 壬戌 \\
    \hline 21-23 & 亥 & 乙亥 & 丁亥 & 已亥 & 辛亥 & 癸亥 \\
    \hline
  \end{tabular}
\end{table}
\footnotetext{甲己还加甲,乙庚丙作初。丙辛从戊起,丁壬庚子居。戊癸何方发,壬子是真途。}
