\section{地支}

\subsection{十二地支}
\begin{table}[H]
  \centering
  \caption{十二地支}
  \begin{tabular}{|c|c|c|c|c|c|c|c|c|c|c|c|c|}
    \hline & 子 & 丑 & 寅 & 卯 & 辰 & 巳 & 午 & 未 & 申 & 酉 & 戌 & 亥 \\
    \hline 属象 & 鼠 & 牛 & 虎 & 兔 & 龙 & 蛇 & 马 & 羊 & 猴 & 鸡 & 狗 & 猪 \\
    \hline 月份 \footnotemark & 冬月 & 腊月 & 正月 & 二月 & 三月 & 四月 & 五月 & 六月 & 七月 & 八月 & 九月 & 十月 \\
    \hline 节 \footnotemark & 大雪 & 小寒 & 立春 & 惊蛰 & 清明 & 立夏 & 芒种 & 小暑 & 立秋 & 白露 & 寒露 & 立冬 \\
    \hline 气 \footnotemark & 冬至 & 大寒 & 雨水 & 春分 & 谷雨 & 小满 & 夏至 & 大暑 & 处暑 & 秋分 & 霜降 & 小雪 \\
    \hline 季节 & \multicolumn{2}{c|}{冬} & \multicolumn{3}{c|}{春} & \multicolumn{3}{c|}{夏} & \multicolumn{3}{c|}{秋} & 冬 \\
    \hline 时辰 & 0:00 & 2:00 & 4:00 & 6:00 & 8:00 & 10:00 & 12:00 & 14:00 & 16:00 & 18:00 & 20:00 & 22:00 \\
    \hline 五行 & 水 & 土 & \multicolumn{2}{c|}{木} & 土 & \multicolumn{2}{c|}{火} & 土 & \multicolumn{2}{c|}{金} & 土 & 水 \\
    \hline 五运 & 君火 & 土 & 相火 & 金 & 水 & 木 & 君火 & 土 & 相火 & 金 & 水 & 木 \\
    \hline 六气 & 热 & 湿 & 火 & 燥 & 寒 & 风 & 热 & 湿 & 火 & 燥 & 寒 & 风 \\
    \hline
  \end{tabular}
\end{table}
\addtocounter{footnote}{-3}
\stepcounter{footnote}\footnotetext{口诀:正寅二卯三见辰,四巳五午六未跟,七申八酉九戌后,十亥逢子十二丑。}
\stepcounter{footnote}\footnotetext{口诀:春雨惊春清谷天,夏满芒夏暑相连,秋处露秋寒霜降,冬雪雪冬小大寒。}
\stepcounter{footnote}\footnotetext{同上}

\subsection{地支六冲}
\begin{table}[H]
  \centering
  \caption[]{地支六冲\footnotemark}
  \begin{tabular}{|c|c|c|c|c|c|}
    \hline 子 & 丑 & 寅 & 卯 & 辰 & 巳 \\
    \hline 午 & 未 & 申 & 酉 & 戌 & 亥 \\
    \hline
  \end{tabular}
\end{table}
\footnotetext{口诀:子午卯酉两相攻,寅申巳亥四气冲,辰戌对开分丑未,地支六冲此中逢。}

\subsection{地支六合}
\begin{table}[H]
  \centering
  \caption[]{地支六合\footnotemark}
  \begin{tabular}{|c|c|c|c|c|c|}
    \hline 子 & 寅 & 卯 & 辰 & 午 & 巳 \\
    \hline 丑 & 亥 & 戌 & 酉 & 未 & 申 \\
    \hline 土 & 木 & 火 & 金 & 太和 & 水 \\
    \hline
  \end{tabular}
\end{table}
\footnotetext{口诀:子丑填土寅亥木,卯戌生火辰酉金,午未太和巳申水,地支二合此中推。}


\subsection{地支三合}
\begin{table}[H]
  \centering
  \caption[]{地支三合\footnotemark}
  \begin{tabular}{|c|c|c|c|}
    \hline 寅午戌 & 亥卯未 & 申子辰 & 巳酉丑 \\
    \hline 火 & 木 & 水 & 金 \\
    \hline
  \end{tabular}
\end{table}
\footnotetext{口诀:申子辰合在水乡,亥卯未遇木相当,寅午戌合将逢火,巳酉丑见藏金匣。}

\subsection{地支三会}
\begin{table}[H]
  \centering
  \caption[]{地支三会\footnotemark}
  \begin{tabular}{|c|c|c|c|}
    \hline 寅卯辰 & 巳午未 & 申酉戌 & 亥子丑\\
    \hline 东 & 南 & 西 & 北 \\
    \hline
  \end{tabular}
\end{table}
\footnotetext{口诀:东方三会寅卯辰,南方三会巳午未,西方三会申酉戌,北方三会亥子丑。}
