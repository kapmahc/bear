\section{唯识}
\subsection{印顺}
\begin{quote}
  好在關於大乘唯識的思想,我在《印度之佛教》(第十四、十五章),《攝大乘論講記》等,已陸續有過簡略的提示了。
\end{quote}

\subsection{真心派}
從印度東方(南)的大眾分別說系發展而來。
真心派重於經典,都編集為經典的體裁:重直覺,重綜合,重理性,重本體論; 以真常心為染淨的所依,雜染是外鑠的。
\footnote{真心論已融化於唯心的大乘經中;妄心者承認大乘經是佛說,即沒有資格去動搖真心論。}

\subsection{妄心派}
從印度西方(北)的說一切有系中出來
妄心派重於論典,如無著、世親等的著作:重思辨,重分析,重事相,重認識論;以虛妄心為染淨的所依,清淨法是附屬的。
\footnote{妄心論的根本論,是未來佛彌勒說,加上嚴密的思辨,真心者也無力摧毀他。}

\subsection{原始佛教}
佛滅一百年以後,佛教才開始顯著的分化。
\footnote{稱這分化了的佛教為部派佛教,分化以前的佛教為原始佛教。}
依據《阿含》和《毘奈耶》(律)。
佛經的正式用文字寫出,在阿育王以後;所用的文字,又有種種的不同。
