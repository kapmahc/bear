\section{十三宗}

\subsection{俱舍宗}
除以俱舍論為正所依外,所宗經有四阿含
\footnote{四阿含者,一長阿含,文言長廣。二中阿含,非略非廣。三增一阿含,於一法上增一至多。四雜阿含,相雜明義。}
\footnote{}
等,論有七論婆沙
\footnote{說六為足,發智為身。}
,及阿毘曇心論,雜阿毘心論等。
\paragraph{二十部執}
先分成大众部、上坐部,后来辗转出二十部。
\paragraph{说一切有部}
於中有曰說一切有,佛滅後三百年初,從上座流出。此部先弘對法,後宏經律。
\paragraph{經量部}
四百年出,說一切有復出一部,名經量部。此部唯依經為正量,不依律及對法。
\paragraph{大毘婆沙論}
三百年末,迦多衍尼子論師,集初期有不論六足論義,造發智論,組織有部之教義。四百年初,有部諸論師,造大毘婆沙論,廣釋發智,大成有部之教義。婆沙出後,有部宗徒,嫌其太博,撮錄其要義綴為專書者層出,就中最簡明而得要者,五百年中法勝論師之阿毘曇心論,以心論太略,造雜阿毘曇心論增補之。
\paragraph{俱舍论}
九百年頃,世親論師出世,初於有部出家,受持有部三藏,後學經部,見其當理,增訂雜心,造俱舍論,述一切有義,時以經部正之,理為長宗,不偏一部。印度學徒,咸稱為總明論。有部眾賢論師,造俱舍雹論,破俱舍。世親見之,為改題為順正理論。一千年頃,安慧無師,糅雜集論,救俱舍,破正理師,遂入大乘。
\paragraph{位置}
大乘諸宗,判釋一代佛教,立有空中三時教之法相宗,置之於有時。立藏通別圓四教之天台宗。置之於藏教。立小始終頓圓五教之華嚴宗,置之於小教。

\paragraph{色法}
五根
\footnote{扶塵根者,扶持助成勝義五根義,其體則麤顯之四塵所成之肉體也。}
\footnote{勝義根者,感取外界之現象,令內界之心識發動之官能。其體淨妙,如珠寶光。}
+五尘
\footnote{火能壞初禪,水能壞二禪,風能壞三禪。}
+无表色

\paragraph{心法}
集起故名心,思量故名意,了別故名識。
\footnote{眼等五識,唯有一自性分別,故名無分別。}
\footnote{第六意識具三,廣考察思惟,故名有分別如論云,分別略有三種,一自性分別,二計度分別,三隨念分別。}

\paragraph{心所法}
\begin{itemize}
  \item 大地法—受 想 思 觸 欲 慧 念 作意 勝解三摩地\footnote{一切心作用生起時,此心所必與之相應俱起}
  \item 大善地法—信 不放逸 輕安 捨 慚 愧 無貪 無瞋 不害 勤
  \item 大煩惱地法—癡 放逸 懈怠 不信 惛沉 掉舉
  \item 大不善地法—無慚 無愧
  \item 小煩惱地法—忿 覆 慳 嫉 惱 害 恨 諂 誑 憍
  \item 不定地法—尋 伺 睡眠 惡作 貪瞋 慢疑
\end{itemize}

\paragraph{心不相應法}
得 非得 同分 無想果 無想定 滅盡定 命根 生 住 異 滅 文身 名身 句身

\paragraph{無為法} 虛空無為 擇滅無為  非擇滅無為
