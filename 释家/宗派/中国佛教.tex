\section{中国佛教}

\subsection{宗派}
中國有「俱舍」、「成實」、「律」、「三論」、「涅槃」、「地論」、「淨土」、「禪」、「攝論」、「天臺」、「華嚴」、「法相」、「密」等十三宗。
其後併涅槃於天臺,併地論於華嚴,併攝論於法相,乃成十宗。
其中,「俱舍」與「成實」為小乘,大乘凡八宗云。
太虛大師曾說:各宗各派的成立,皆由古代祖師依其修行經驗為主,適應當時當地的眾生機宜而成立的。
\footnote{《佛法是救世之光》}

\begin{quote}
  三界唯心,萬法唯識之旨,受有部之影響特深。
\end{quote}
