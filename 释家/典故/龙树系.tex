\section{龙树系}

\subsection{提婆菩薩}
如同龍樹一樣,提婆也感化了一位本來信奉外道的國王,又以辯論的方式,摧破了外道,三月之間,度百餘萬人。有一外道的弟子,因其師遭提婆論破,懷忿在心,誓言:「汝以囗勝伏我,我當以刀勝伏汝;汝以空刀困我,我以實刀困汝。」於是,有一天,提婆菩薩正在閑林經行,這個外道弟子便捉刀而至,並說:「汝以囗破我師,何如我以刀破汝腹!」
提婆菩薩,雖已腹破而五臟落地,但仍哀憫此一兇手的愚癡,告訴他說:「吾有三衣鉢盂,在吾住處,汝可取之,急上山去,勿就平道。我諸弟子未得法忍者,必當捉汝,或當相得送汝於官。」

當弟子們趕到現場,有些未得法忍的人,便大哭大叫,狂突奔走,要追截兇手。提婆菩薩反而藉此機緣向弟子們開示:「諸法之實,實無受者,亦無害者,誰親誰怨,誰賊誰害?汝為癡毒所欺,妄生著見而大號咷,種不善業。彼人所害,害諸業報,非害我也。汝等思之,慎無以狂追狂,以哀悲哀也。」告誡完畢,他便放身蟬蛻而去。
