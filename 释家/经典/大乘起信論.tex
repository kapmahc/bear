\section{大乘起信論}
本論以「法」為「眾生心」,法是大乘法,眾生心即是如來藏;換言之,如來藏便是大乘法。眾生心含攝一切法,故名「一心」,由此一心向清淨界、光明界、悟界看,便是真如門;由此一心向雜染界、無明界、迷界看,便是生滅門。真如門是自性清淨心,生滅門是雜染虛妄心;由無明而有虛妄生滅,由虛妄生滅的現實而修習向上,即可至究竟的果位,稱為一心法界。這一心法界在本體、功能、作用的三方面,即稱為三大:
\begin{enumerate}
  \item 體大(本體),即是不生、不滅、不垢、不減的真如實性。
  \item 相大(功能),即是真如含有無限的德相。
  \item 用大(作用),即是能生世出世間之無漏有漏的一切善法。
\end{enumerate}
\textbf{一心、二門、三大},乃是《大乘起信論》的綱骨。

\paragraph{真如}
是古人對於宇宙本體的命名,它是遠離了一切妄念後的實在心,一切法無不是真如。本論對於真如的界別,分有離言真如及依言真如、空真如及不空真如,著眼點是在不空真如之含攝無量功德,那就是常樂我淨的一心,也是我們修學佛法的目標。但此一心,眾生本具,悟時即可見此自性清淨的真如心了。

\paragraph{生滅}
是由眾生心開展出來的現象界,它與真如對立,稱清淨的如來藏為真如,呼「不生滅」及「生滅和合」的最初狀態為阿黎耶識(新譯為阿賴耶識)。
因了阿黎耶識而有無明,因了根本無明而有枝末無明,此無明的相狀即是心的活動。本論將無明的流轉,立三細六麤之說。所謂三細六麤,是統括生滅門的九種心的活動程序,稱為九相,此九相含攝了十二緣起的內容,也就是十二緣起的新的特別解釋法。不過,十二緣起是就外在的、是明此一身流轉三界的順序而說,本論的九相是專就內部的、是明此一心變化所行的順序而說。約眾生心識的次第開展而說,分為心、意、意識。約眾生心的惑障而說,又分為無明與染心。
總之,本論所強調的是一心,這個心要比唯識家的識,更加堅強,迷悟不離一心,大乘法即是眾生心。換言之,信佛、學佛、成佛,也都是信仰我們自己的心,學習我們自己的心,成就我們自己的心;我們的現象界是由我們的心所促成,我們的本體界也要由我們對自己的心來開發。所以這在哲學上說,乃是絕對的唯心論。

\begin{quote}
  《華嚴經》開出淨心緣起的花,《大乘起信論》是最後結成的果;中國的傳統佛教,就是沿著這一條路在走,無怪乎覺得《大乘起信論》是如此的重要。
\end{quote}
