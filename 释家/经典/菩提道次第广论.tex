\section{菩提道次第广论}

\paragraph{西藏佛教传承}
靜命大論師\footnote{西藏的正式有三寶}
蓮花生大士\footnote{大弘教法}
阿底峽尊者\footnote{从朗達瑪灭佛复兴,改革之后称为新教}\footnote{《阿底峽尊者傳》}
宗喀巴大師\footnote{不拘任何宗派,小乘的《俱舍》,然後呢,大乘的性宗、相宗、因明,乃至密教,然後自己有圓滿徹底的認識}
\footnote{彻底改革}
\footnote{《應化因緣集》}
太虛大師\footnote{八宗並弘, 大勇法師->法尊法師}

\begin{quote}
  你對整個的教法有了圓滿的認識,把你的目標確定好了以後,那時候你進一步選你現在相應,應該走的路。
\end{quote}
\paragraph{學哪一個宗派}
一開頭的時候,我們要了解佛法的整個圓滿的內容是什麼,你有了正確的了解了以後,然後把你的目標一開始的時候,規劃出來。從這一個認識當中,選取找到你自己相應的路,然後你走上去的話,千穩百當,而且是最省事、最快速、最圓滿;念佛照樣地念,參禪照樣地參。



\subsection{道前基础}
\paragraph{觉}自覺、覺他、覺行圓滿
\begin{quote}
  质\footnote{要正} vs 量\footnote{要足}
\end{quote}
\begin{quote}
  講得頭頭是道,做起來是一無是處
\end{quote}
\paragraph{通常說圓教經典}一部是最初說的《華嚴》,一部是最後說的《法華》\footnote{《化城喻品》}
\paragraph{要真正成佛,要做兩件事情}所知障徹底地淨除,煩惱障徹底淨除;「廣」是無所不包,「深」是徹見本源
\paragraph{三身}
菩薩看見他的是報身;眾生還沒有登地之前,還沒有破無明之前,看見他的是化身,而這個化身有在四生、六道當中的。
\paragraph{智慧、方便}
彌勒菩薩代表方便(广行)\footnote{阿逸多},文殊菩薩代表智慧(深观)\footnote{妙音}
\paragraph{大悲}
菩提心就是發救一切眾生心這個願力
\paragraph{廣行}
用種種方式,種種方便去幫助別人
\begin{quote}
  菩薩清涼月,遊於畢竟空
\end{quote}
\begin{quote}
  如理、如量、次第
\end{quote}
\paragraph{瑜伽}分成境、行、理、果
\paragraph{因明}陳那、法稱



\subsection{下士道}
指出凡夫轮回受苦的原因\textbf{无明、迷、不觉}。

\subsubsection{念死无常}
\subsubsection{三恶趣苦}
\subsubsection{皈依三宝}
\begin{quote}
  信为能入,智为能度
\end{quote}
\begin{quote}
  分分断证
\end{quote}
\subsubsection{深信因果}
第一义愚\footnote{不解性空之理},业果愚\footnote{不解缘起相循的因果关系}
\begin{quote}
  性空所以緣起,緣起所以性空
\end{quote}
\begin{quote}
  寧取有見如須彌山,不取空解如芥子許
\end{quote}




\subsection{中士道}
\begin{quote}
  集諦為因,苦諦是果,道諦是因,滅諦是果。
\end{quote}
\subsubsection{思惟苦諦}
\begin{quote}
  三種雜染:惑、業、苦
\end{quote}
\begin{quote}
  苦苦\footnote{遇到苦的事當然苦}、
  壞苦\footnote{遇到快樂的事耽溺於樂受中,但這種感受既不真實又不長久,當樂受變壞時便痛苦無比}、
  行苦\footnote{遇到不苦不樂的捨受中,因為遷流變動,無常所隨逐,感到不安穩}
\end{quote}
\subsubsection{思惟集諦}
\subsubsection{思惟十二有支}
\subsubsection{思惟解脫生死正道}



\subsection{上士道}
\begin{quote}
  為救度一切眾生,自己必須成就無上佛果。由此發起菩提心,求受菩薩戒,學習六度,修學四攝
\end{quote}
\subsubsection{發大菩提心}
\subsubsection{修菩薩行}


\subsection{止观}
\begin{quote}
  止觀兼修才能明瞭見真實義,詳細抉擇清淨正見
\end{quote}
