\section{密教史}

\subsection{栂尾祥雲}

根據古來傳統的說法,釋尊滅後八百年,有龍樹大士開南天竺之鐵塔,誦出藏於塔中的祕密大經,即成為密教的紀元。

\paragraph{佛陀禁制密法}
根據《中阿含經》、《長阿含經》,以及《四分律》等記載,最初佛弟子們,是嚴禁行使世俗的咒術密法的,如果破壞了這項規定,便是犯了波逸提(pāyattika)罪。
尤其在巴利經典的《小品(Cullavagga).小事篇》第五,把世俗的密法, 彈訶成為浮牲之學(Tirachana-Vijjā)。
若依脫俗為宗旨的佛的根本立場而言,根本沒有可能為了治病、延命、招福等欲樂利益,而使用咒術密法等的餘地。

\paragraph{攝取密法的實情}
弘佈佛教是為了普遍攝收所有各方面的群眾,故在攝化的方便上,對這些人的生活習慣所關聯的行事信仰,也必須予以調和、淨化、疏導。於是,到了羅什三藏所譯的《十誦律》卷四六等,所說的那樣,對於妨害修行佛道的惡咒密法,當然嚴禁,至於像治毒咒、治齒咒,以及守護一身、自得安慰的善咒,不妨誦持。這也即是認可了咒術密法的地位。
