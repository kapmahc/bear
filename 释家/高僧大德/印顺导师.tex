\section{印顺导师}

\subsection{妙雲集}
\subsubsection{上編}
經論的解說
\begin{itemize}
  \item 《般若經講記》\footnote{這包含了《金剛經》及《心經》的兩部講記}
  \item 《寶積經講記》
  \item 《勝鬘經講記》
  \item 《藥師經講記》
  \item 《中觀論頌講記》
  \item 《攝大乘論講記》
  \item 《大乘起信論講記》
\end{itemize}
\subsubsection{中編}
是專著而篇幅超過十萬字的
\begin{itemize}
  \item 《佛法概論》
  \item 《中觀今論》
  \item 《唯識學探源》
  \item 《性空學探源》
  \item 《成佛之道》
  \item 《太虛大師年譜》
\end{itemize}
\subsubsection{下編}
是短篇的總集,不問是寫的,記錄的,都編在一起。
\begin{itemize}
  \item 《佛在人間》\footnote{重於人間佛教的現實利益,從人乘正行而向佛道。}
  \item 《學佛三要》\footnote{信願、智慧、慈悲——為大乘佛法的三要,學者要不偏不離的去學習。}
  \item 《以佛法研究佛法》
  \item 《淨土與禪》
  \item 《青年的佛教》
  \item 《我之宗教觀》
  \item 《無諍之辯》
  \item 《教制教典與教學》
  \item 《佛教史地考論》
  \item 《華雨香雲》
  \item 《佛法是救世之光》
\end{itemize}
