\section{四}

\subsection{四諦}
\begin{itemize}
  \item 苦諦:人生如苦海。
  \item 集諦:集是苦的原因,由煩惱而造業,由造業而招感苦的果報。
  \item 滅諦:滅是解脫苦果的可能,明瞭集諦之理,斷除煩惱之業,即可解脫眾苦。
  \item 道諦:道是滅苦的方法,修持八正道,即可滅除眾苦而獲涅槃解脫之果。
\end{itemize}

\subsection{四念住}
主要對治執身為凈、執受為樂、執心為常、執法為我的“四顛倒見”。
\begin{itemize}
  \item 身念住,觀身不凈
  \item 受念住,觀受是苦
  \item 心念住,觀心無常
  \item 法念住,觀法無我
\end{itemize}

\subsection{四正勤}
精進的重點在於行善去惡。
\begin{itemize}
  \item 未生惡法令不生;
  \item 已生惡法恒令滅;
  \item 未生善法令出生;
  \item 已生善法令增長。
\end{itemize}


\subsection{四神足}
意為產生精進的基礎
\begin{itemize}
  \item 欲神足,欲得見道;
  \item 勤神足,精勤習禪;
  \item 心神足,心神專一;
  \item 觀神足,正確觀想。
\end{itemize}


\subsection{四暴流}
\begin{itemize}
  \item 欲暴流
  \item 有暴流
  \item 見暴流
  \item 無明暴流
\end{itemize}

\subsection{四漏}
\begin{itemize}
  \item 欲漏
  \item 有漏
  \item 見漏
  \item 無明漏
\end{itemize}

\subsection{四取}
\begin{itemize}
  \item 欲取  \item 見取  \item 戒禁取  \item 我語取
\end{itemize}

\subsection{四繫}
\begin{itemize}
  \item 貪繫  \item 瞋繫  \item 戒禁取繫  \item 是真執繫
\end{itemize}
