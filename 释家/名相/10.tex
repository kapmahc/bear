\section{十}

\subsection{《攝大乘論》十種殊勝相}
十種殊勝相可分為境、行、果的三類。一及二是境殊勝,三至八是行殊勝,九及十是果殊勝。
\begin{itemize}
  \item 所知依:「謂阿賴耶識,說名所知依體」。一切所應知的法,都依於阿賴耶識而立。也就是說,阿賴耶識為三性(遍計執、依他起、圓成實)所依。對三性有兩種見解:遍計執及依他起是雜染,圓成實是清淨。遍計執是雜染,圓成實是清淨,依他起通於雜染和清淨之二邊。照第一見解說,阿賴耶是虛妄不實、雜染不淨;照第二見解說,阿賴耶既是虛妄雜染,也是真實清淨。無著側重在第一種見解,世親則兼談兩種。
  \item 所知相:「三種自性」,「說名所知相體」。所知就是相,名為所知相。即是將一切所應知的法,分為三相來說明:1.依他起自性─是仗因托緣而生起的、可染可淨而不是一成不變的一切法。2.遍計所執自性─是指亂識(幻妄)所取的一切法,毫無實體,不過是一種自己的錯覺意境。3.圓成實自性─是指由人空及法空所顯的諸法之真實性。
  \item 入所知相:「唯識性,說名入所知相體」。由修唯識觀而悟入唯識性,就是悟入所知相的真實性。唯識觀有兩種:1.初步的方便唯識觀─以唯識觀觀一切法的自性為虛妄分別,所以了不可得。2.進一步的真實唯識觀─觀察諸法之境不可得,虛妄分別的識也不可得,心境俱泯,即悟入平等法性(圓成實性)。本論的唯識觀雖通於真實境(地上),但重於從凡入聖(由加行分別智到根本無分別智)的唯識觀。
  \item 彼入因果:「六波羅蜜多,說名彼入因果體」。彼入就是入彼,即是說,要悟入彼(那個)唯識的實性,必須修習六種波羅蜜多(布施、持戒、忍辱、精進、禪定、智慧);尚未悟入唯識性時,所修者是因,證入唯識性以後,所修者即是果。
  \item 彼因果修差別:「菩薩十地,說名彼因果修差別體」。進入初地以後的聖位菩薩,於十地中,仍是修習六波羅蜜多;地地增上,故說有十地差別。到佛果時,六波羅蜜多的修習,即告圓滿。
  \item 修差別中增上戒:「菩薩律儀,說名此中增上戒體」。即是諸地之中菩薩所修的戒學,表明他們不是修的聲聞小乘戒,故稱菩薩律儀。地地修習、展轉、增加、向上,故名增上。
  \item 增上心:「首楞伽摩,虛空藏等諸三摩地,說名此中增上心體」。此即是諸地菩薩所修的定學。定以心為主體,故稱增上心。首楞伽摩,義為健行,就是首楞嚴大定,此定境界很高,為十住菩薩所修。虛空藏也是定名,能含攝、能出生一切功德,所以名為虛空藏。
  \item 增上慧:「無分別智,說名此中增上慧體」。此即菩薩所修的慧學。無分別智含有加行智、根本智、後得智三者。菩薩遠離一切法執分別,故此三智皆稱無分別智。
  \item 彼果斷:「無住涅槃,說名彼果斷體」。彼果,就是修習那戒、定、慧三增上學所得的果。那果就是斷煩惱障及所知障而得的果,故稱果斷。無住涅槃是不住於生死也不離於生死之義。
  \item 彼果智:「三種佛身」,「說名彼果智體」。上面所說的那個果,就是智,故名果智。從果的斷障寂滅而言,是無住大般涅槃;從果顯現的智慧而言,是圓滿的無分別智,亦即是八識轉為四智而成就三種佛身:1.第八識轉成大圓鏡智、第七識轉成平等性智,即是「自性身」佛。2.第六識轉成妙觀察智,即是「受用身」佛。3.前五識轉為成所作智,即是「變化身」佛。第一種身是常住的,二、三種身是無常的。由自性身現起的受用身,受用一切法樂(自受用),並為地上的聖位菩薩說法(他受用);由自性身現起的變化身,則為聲聞說法。
\end{itemize}

\subsection{十法行}

\begin{itemize}
  \item 書寫,
  \item 供養
  \item 施他
  \item 諦聽
  \item 披讀
  \item 受持
  \item 開演
  \item 諷誦
  \item 思惟
  \item 修習
\end{itemize}
