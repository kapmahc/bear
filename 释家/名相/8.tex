\section{八}

\subsection{八苦}
苦苦之中又有八種
\begin{itemize}
  \item 生苦
  \item 老苦
  \item 病苦
  \item 死苦
  \item 愛別離苦
  \item 怨憎會苦
  \item 求不得苦
  \item 五蘊熾盛苦
\end{itemize}

\subsection{八正道}
八正道即是四聖諦中的道諦\footnote{由八正道,開演出三十七道品,又歸納演化為六波羅蜜多(六度),但其均屬於戒、定、慧的三無漏學的範圍。}
\begin{itemize}
  \item 正見:即是正確的見解。\footnote{何為正見?則應以三法印來鑑定}。
  \item 正思惟:即是以正見為基礎,而來思量熟慮此正見的內容,這是「意」業的實踐工夫。
  \item 正語:基於正確的意念,表達於「口」業的實踐工夫,不得對人妄言欺騙、綺語淫詞、兩舌挑撥、惡口罵辱,而且要做善言勸勉、愛語安慰。
  \item 正業:即是正當的身業,不做殺生、偷盜、淫亂、使用麻醉物等的惡業。配合意、語二業,即是「身、語、意」的三業清淨。
  \item 正命:即是正當的謀生方法,除了不做惡業,更應以正當職業,謀取生活所需。不得以江湖術數等的伎倆,騙取不義之財。
  \item 正精進:即是策勵自己,努力於道業。惡之尚有未斷者,立即求其斷,善之尚有未修者,立即求其修;未起之惡令不起,已修之善令增長。
  \item 正念:既已有了策勵精進之心,即應攝心制心,以不淨觀等方法,使心住於一境,不起物我之思。
  \item 正定:循著前面的七階段來修持,必可進入四禪八定,最後再以空慧之力,進入滅受想定,便是涅槃的解脫境界。
\end{itemize}
