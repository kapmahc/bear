\section{七}

\subsection{七众弟子}
順著次序等位來說;
通常所謂比丘二百五十戒,比丘尼五百戒,式叉摩尼六法,沙彌及沙彌尼十戒,優婆塞及優婆夷即是在家的男女弟子,有三皈五戒。
七眾的界別,即是根據所受持的戒法而定。
\begin{enumerate}
  \item 比丘(Bhikṣu)。
  \item 比丘尼(Bhikṣuṇī)\footnote{由於摩訶婆闍波提以及釋種五百女子的出家,便有了比丘尼。}。
  \item 式叉摩尼(Śāikṣamāṇā)。
  \item 沙彌(Śrāmaṇera)\footnote{由於少年羅睺羅的出家,僧中即有了沙彌。}。
  \item 沙彌尼(Śrāmaṇerikā)。
  \item 優婆塞(Upāsaka)\footnote{由於頻婆沙羅王的皈依佛教,在家的男女信徒即日漸增加}。
  \item 優婆夷(Upāsikā)。
\end{enumerate}


\subsection{七覺支}
修習止觀的注意事項和感受。
\begin{itemize}
  \item 念覺支,憶念集中而念念分明;
  \item 擇法覺支,選擇正確、適宜的修法;
  \item 精進覺支,任何階段都不能懈怠;
  \item 喜覺支,修禪定得到的喜悅;
  \item 輕安覺支,得到的輕鬆安適感覺;
  \item 定覺支,攝心不散深入禪定;
  \item 捨覺支,捨一切念,不即不離。
\end{itemize}

\subsection{七結(使)}
\begin{itemize}
  \item 欲貪
  \item 有貪
  \item 瞋恚
  \item 慢
  \item 見
  \item 疑
  \item 無明
\end{itemize}
