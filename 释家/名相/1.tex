\section{一}

\subsection{外道}
是佛典中的術語,梵語叫作底他迦(Tīrthaka),是指佛教以外之教道,或稱為外教、外學、外法。
凡是佛教以外的一切教道,以佛子視之,均為外道,初無輕藐之意;然以外教學者無不捨自己的身心而別求安頓,所以含有向外求道的意思者,即為外道。      
