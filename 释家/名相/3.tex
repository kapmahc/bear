\section{三}

\subsection{三藏}
《增一阿含經‧序品》中則加雜藏為四藏,《分別功德論》及《成實論》又將雜藏分為雜藏及菩薩藏,成了五藏。
\begin{itemize}
  \item 經 \item 律 \item 論
\end{itemize}


\subsection{三法印}
即是用三句話來印證諸法,合乎這三句話的標準,便印可它是合於佛法的正見,否則便是魔外偏妄的邪見
\begin{itemize}
  \item 諸行無常
  \item 諸法無我
  \item 涅槃寂靜
\end{itemize}

\subsection{三無漏學}
由持戒清淨之後,修禪才能得正定;由正定的定力,可以產生無漏的慧力;再由慧力來指導持戒。
唯有藉著空慧或無漏慧的正見,持戒才會恰如其分,修禪才不致歧入魔境。
\begin{itemize}
  \item 戒
  \item 定
  \item 慧
\end{itemize}

\subsection{三毒、三不善根、三火}
\begin{itemize}
  \item 贪
  \item 嗔
  \item 痴
\end{itemize}



\subsection{三世}
\begin{itemize}
  \item 过去世
  \item 现在世
  \item 未来世
\end{itemize}

\subsection{三轉四諦}
\begin{itemize}
  \item 示轉:說明此是苦、此是集、此是滅、此是道。
  \item 勸轉:說明苦應知、集應斷、滅應證、道應修。
  \item 證轉:說明苦者我已知、集者我已斷、滅者我已證、道者我已修。
\end{itemize}

\subsection{三苦}
\begin{itemize}
  \item 苦苦
  \item 壞苦
  \item 行苦
\end{itemize}

\subsection{三連鎖}
\includegraphics[scale=0.5]{释家/images/三连锁.png}
\begin{itemize}
  \item 惑:過去世的無明,現在世的愛及取。
  \item 業:過去世的行,現在世的有。
  \item 苦:現在世的識、名色、六入、觸、受,未來世的生及老死。
\end{itemize}



\subsection{三求}
\begin{itemize}
  \item 欲求  \item 有求  \item 梵行求
\end{itemize}
