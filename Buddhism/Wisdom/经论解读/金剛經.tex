\subsection{金剛經}

\subsubsection{经题}
\paragraph{金剛} 坚常、明净、快利
\paragraph{般若}
\begin{itemize}
  \item 实相般若
    \footnote{《智論》說:「般若如大火聚,四邊不可觸」}
    \footnote{《智論》說:「觀是一邊,緣是一邊,離此二邊說中道」}
    \footnote{古德說:「說似一物即不中」}
    \footnote{《法華經》說:「唯佛與佛乃能究盡諸法實相,所謂:諸法如是相,如是性,如是體,如是力,如是作,如是因,如是緣,如是果,如是報,如是本末究竟等」}
    \footnote{《中論》說:「空則不可說,非空不可說,共不共叵說,但以假名說」}
    \footnote{《中論》說:「一切實非實,亦實亦非實,非實非非實,是名諸佛法」}
  \item 观照般若
    \footnote{《智論》說:「般若是一法,隨機而異稱」}
    \footnote{《智論》說:「未成就名空,已成就名般若」}
    \footnote{「因名般若,果名薩婆若」}
    \footnote{《智論》這樣說:「般若將入畢竟空,絕諸戲論;方便將出畢竟空,嚴土熟生」。}
    \footnote{智論》比喻說:般若如金,方便如熟煉了的金,可作種種飾物。}
  \item 文字般若
\end{itemize}
