\section{学习方法}

\begin{quote}
  修學佛法,以正見為先。依正見(聞思慧)而起正信,依正見而修戒、定,最後以(現證) 慧得解脫。
  \footnote{在研求的態度上,應有「無我」的精神。「無我,是離卻自我(神我)的倒 見,不從自我出發去攝取一切。在佛法的研究中,就是不固執自我的成見,不( 預)存一成見去研究」,讓經論的本義顯現出來。「切莫自作聰明,預存見解, 也莫偏信古說」。}
\end{quote}

\begin{quote}
  「非精嚴不足以圓融」
\end{quote}

\begin{itemize}
  \item 「從論入手」
  \item 「重於大義」
  \item 「重於辨異」
  \item 「重於思惟」
\end{itemize}

\begin{quote}
  道非經無以寓,法非經無以傳。緣經以求法,緣法以悟道,方識是經之旨清淨微妙第一希有。
\end{quote}

\begin{quote}
  「 方便有多門,歸元無二路」
\end{quote}

\begin{itemize}
  \item 有以信樂十方淨土精進而入佛道的,是信增上人
  \item 有以智 慧解悟而入佛道的,是智增上人
  \item 有以悲濟眾生而入佛道的,是悲增上人。
\end{itemize}
