\section{基础方法}

\subsection{七支坐法}
\begin{enumerate}
  \item
\end{enumerate}

\subsection{靜坐、禪定與禪}
\paragraph{靜坐} 只能使你身體鬆弛,頭腦輕安
\paragraph{禪定} 能使你達到身心統一,最高的境界則是前念與後念,念念統一,但不能把自我中心的念頭放棄。
\paragraph{禪} 是要放棄定境後,無我的智慧自然出現。

\subsection{原则}
\begin{itemize}
  \item 身心放松
    \footnote{出现身心反应时不去管它}
  \item 观、照、提
  \item 安定清明
\end{itemize}
\begin{quote}
  以路標為目的地是愚癡,不依路標所指而前進,更加危險;
\end{quote}


\subsection{次第}
\begin{enumerate}
  \item 散乱心
  \item 集中心
  \item 统一心
  \item 无心
\end{enumerate}

\subsection{日常应用}
\begin{itemize}
  \item 身心合一
  \item 心口一致
  \item 心眼一如
\end{itemize}
