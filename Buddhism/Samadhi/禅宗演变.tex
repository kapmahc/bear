\section{禅宗演变}

\subsection{菩提達摩}
二入四行
\begin{itemize}
  \item 理入
  \item 行入
\end{itemize}
\begin{itemize}
  \item 報冤行
  \item 隨緣行
  \item 無所求行
  \item 稱法行
\end{itemize}
\subsection{四祖道信}
《入道安心要方便門》
\begin{quote}
  「內外空淨,即心性寂滅」
\end{quote}
\subsection{五祖弘忍}
〈修心要論〉(即〈最上乘論〉)
\begin{quote}
  「端坐正念,閉目合口,心前平視,隨意近遠,作一日想,守真心,念念莫住。」
  \footnote{《觀無量壽經》:「有目之徒皆見日沒,當起想念,正坐西向,諦觀於日,令心堅住。」}
\end{quote}
\begin{quote}
  「但知攝心莫著,並皆是空」
\end{quote}
\begin{quote}
  「十方國土,皆如虛空,三界虛幻,唯是一心作。」
\end{quote}
\subsection{六祖惠能}
\begin{quote}
  「不思善、不思惡」
\end{quote}
\begin{quote}
  「無念」\footnote{面對內外善惡境界之時,心中不起一絲波動。}
  、「無住」\footnote{「住」是執著之意}
  、「無相」\footnote{《金剛經》所示: 無我相、無人相、無眾生相、無壽者相}
\end{quote}
\subsection{參公案}
不是猜測揣摩,不是用頭腦推敲思索,不能用常識及佛學的知識來解釋它。
\footnote{南泉斬貓}
\subsection{長蘆宗賾禪師}
《坐禪儀》
\begin{quote}
  「量其飲食,不多不少」
  「調其睡眠,不節不恣」
  「欲坐禪時,於閑靜處」
  「放捨諸像,休息萬事,身心一如,動靜無間」
  「一切善惡都莫思量,念起即覺,覺之即失,久久忘緣,自成一片。」
\end{quote}
\footnote{日本的道元禪師,依據它寫了一篇〈普勸坐禪儀〉}
