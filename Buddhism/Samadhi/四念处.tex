\section{四念处}
佛教中很基本而且重要的方法
\footnote{释尊將入涅槃時,咐囑阿難:經文皆以「如是我聞」為起始;以戒律為師;依四念處行道;默擯犯戒而不受勸告的比丘。}
,与五停心
\footnote{五停心屬於奢摩他,是止;四念處屬於毘婆奢那,即觀。}
关系密切。

\subsection{原理}
五停心最主要的作用,是針對散心和亂心的人,使他們能夠循著方法把心集中起來,漸漸地達到定的程度。
修五停心得定以後,立即從定出來,用有漏智慧觀察四念處的身、受、心、法,一直觀想,進而達到發無漏慧、出三界的目的。
最主要的差別在於修五停心可以得定,但不能開悟,在定中心亦無法修;必須是得定之後,從定出,以有漏慧來修四念處,從四念處發無漏慧。
\footnote{有定有止,再修觀,從觀發慧,這個與天臺宗所講的止觀均等不同。}
\footnote{如果不得定便修四念處,或是修五停心得定後許久,定力退失,才修四念處,都是不對的。}
觀四念處得力,才知道四聖諦的苦諦究竟是什麼。
能夠瞭解到苦諦的實義,才能真切地生起不退的信心,直到證四聖諦,已能每一剎那、每一剎那,都連續地、不斷地、不退地觀照明徹,而且就在觀上面不會再離開了。
故證四聖諦已具無漏慧,出離三界,永斷生死。

\subsection{方法}
對治四種顛倒\footnote{常乐我净}的錯認和執著。
\begin{itemize}
  \item 觀身不淨
  \item 觀受是苦
  \item 觀心無常
    \footnote{不是不許心念活動,而是明察心念在做什麼。}
  \item 觀法無我
\end{itemize}

\subsection{四加行}
並要配合十六特勝
\footnote{目的是要證四聖諦:知苦、斷集、證滅、修道。}
\footnote{四念处*四圣谛=十六特胜}
來修
\begin{itemize}
  \item 煖
  \item 頂
  \item 忍
  \item 世第一
\end{itemize}
