\section{五停心}

\subsubsection{數息观}
\paragraph{方法}
\begin{itemize}
  \item 先感覺呼吸是從鼻孔出入
  \item 数出息,不数入息
  \item 心念就貼在數目上
    \footnote{不可有數目「字」的形象}
  \item 不要經常去注意呼吸
    \footnote{僅僅是在呼吸時,知道鼻孔有氣息的出入,然後就不注意鼻孔,而將注意力轉移到數目上,將心念貼在數目上。}
  \item 数错就回到一
\end{itemize}
\paragraph{常见错误}
\begin{itemize}
  \item 用头脑想
  \item 控制呼吸,以防雜念出現
  \item 呼吸忽長忽短
  \item 閉氣
  \item 腹部肌肉緊張
\end{itemize}
\paragraph{控制呼吸的解决办法}
\begin{enumerate}
  \item 调身:逐步放松
  \item 调心:平静
  \item 专注呼吸
\end{enumerate}


\subsubsection{不淨观}

\subsubsection{慈心观}

\subsubsection{界方便观}

\subsubsection{因緣观}
