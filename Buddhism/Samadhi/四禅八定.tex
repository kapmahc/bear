\section{四禅八定}

\subsection{禅}
静\footnote{寂静}虑\footnote{观察}
唯独色界天的四禅才可以叫做禅,叫做静虑的。
\footnote{未到地定、欲界定力弱;无色界定力过强;唯有色界可以均等}
\subsection{定}
等持:
平等持心\footnote{不昏沉;不散乱}
于一境转\footnote{在一个境界上相续而住}
有所成办\footnote{成就一切你要成就的事情}



\subsection{四果}
\begin{itemize}
  \item 須陀洹
    \footnote{預入法流, 截斷了生死的根源(三結),成為聖者。即使修行遲緩或停 頓,也不會墮入惡趣;再多也不過七番生死,一定要到達究竟解脫的。}
  \item 斯陀含
    \footnote{「一來」,再多也只有人間、天上──一番生死了。}
  \item 阿那含
    \footnote{「不來」, 證得阿 那含果的,如死後上生,就在天上入涅槃,不會再來人間了。}
  \item 阿羅漢
    \footnote{有應(受尊敬供養) ,殺賊,不生等意義。這是究竟解脫聖者的尊稱,依修道求解脫來說,這是最究竟的}
    \begin{itemize}
      \item 法住智:緣起法被稱為「 法性」、「法住」,知法住是知緣起。從因果起滅的必然性中,於(現實身心)蘊、界、處如實 知,厭、離欲、滅,而得「我生已盡,梵行已立,所作己辦,不受後有」的解脫智。雖沒有根本 定,沒有五通,但生死已究竟解脫,這是以慧得解脫的一類。
      \item 涅槃智:  是慧解脫者的末「後知涅槃」;也有生前得見法涅槃,能現證知涅槃, 這是得三明、六通的,名為(定慧)俱解脫的大阿羅漢。
    \end{itemize}
    \footnote{雖有二類 不同,但生死的究竟解脫,是一樣的;而且都是「先知法住,後知涅槃」的。}
\end{itemize}
禪定是共世間法,即使修得非想非非想定,也不能解脫生死;反而不得根本定的,也能成慧解脫阿羅漢。


\subsection{欲界定}
粗住\footnote{一盘腿两小时,明而静,静而明}
细住\footnote{继续精进,增强定力;身体感觉若存若亡}
\subsubsection{持身法}
会自然出现,会让身体自然端正而坐,很轻快


\subsection{未到地定}
只一念明了这一念心在,身体内外境不在了. 

\subsection{色界定}
\subsection{无色界定}
