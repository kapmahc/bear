\chapter{参考书籍}

\begin{itemize}
  \item CBETA 中华电子佛电协会 \url{http://www.cbeta.org/download/cbreader.htm}
  \item 圣严法师$\cdot$法鼓全集 \url{http://ddc.shengyen.org/mobile/}
  \item 印順法師佛學著作集 \url{http://yinshun-edu.org.tw/Master_yinshun/books}
  \item 太虚大师全集
\end{itemize}

\section{福嚴精舍三年閱經目錄}
印順導師擬
\subsection{第一年}
三百餘卷,其中包括從印度譯來的經、律、論,大乘、小乘、空宗、有宗等各樣代表典籍。
\subsubsection{甲}
\begin{itemize}
  \item 五分律(受比丘戒者):《彌沙塞部和醯五分律》 (30卷) 【劉宋 佛陀什共竺道生等譯】
  \item 學佛行儀(未受比丘戒者閱讀):\url{http://www.bfnn.org/book/books2/1265.htm}
  \item 五戒相經(未受比丘戒者):《佛說優婆塞五戒相經》(未受比丘戒者) (1卷) 【宋 求那跋摩譯】
  \item 菩薩優婆塞戒經(未受比丘戒者):《優婆塞戒經》 (7卷) 【北涼 曇無讖譯】
  \item 四十二章經(未受比丘戒者):《四十二章經》 (1卷) 【後漢 迦葉摩騰共法蘭譯】
  \item 雜阿含經 (50卷):《雜阿含經》 (50卷) 【劉宋 求那跋陀羅譯】
\end{itemize}
\subsubsection{乙}
\begin{itemize}
  \item 大涅槃經初分(南本初十卷):《大般涅槃經》(南本初十卷) (36卷) 【宋 慧嚴等依泥洹經加之】
  \item 大涅槃經初分(北本初十一卷):《大般涅槃經》(北本初十一卷) (40卷) 【北涼 曇無讖譯】
  \item 華嚴經十地品:《大方廣佛華嚴經》(十地品) (80卷) 【唐 實叉難陀譯】
  \item 法華經:《妙法蓮華經》 (7卷) 【姚秦 鳩摩羅什譯】
  \item 楞嚴經(十卷):《大佛頂如來密因修證了義諸菩薩萬行首楞嚴經》 (10卷) 【唐 般剌蜜帝譯】
  \item 大般若經第四分(五三八~五五五卷):《大般若波羅蜜多經(第401卷-第600卷)》 (200卷) 【唐 玄奘譯】
  \item 小品般若波羅蜜經:《小品般若波羅蜜經》 (10卷) 【後秦 鳩摩羅什譯】
  \item 大集經海慧品:《大方等大集經》 (60卷) 【北涼 曇無讖譯】
  \item 寶積經不動如來品(菩薩藏會、富樓那會、摩訶迦葉會、無盡菩薩會、無量壽如來會):《大寶積經》 (120卷) 【唐 菩提流志譯】
  \item 佛藏經:《佛藏經》 (3卷) 【姚秦 鳩摩羅什譯】
  \item 大樹緊那羅王所問經:《大樹緊那羅王所問經》 (4卷) 【姚秦 鳩摩羅什譯】
  \item 寶雲經:《寶雲經》 (7卷) 【梁 曼陀羅仙譯】
  \item 無上依經:《佛說無上依經》 (2卷) 【梁 真諦譯】
  \item 諸佛要集經:《諸佛要集經》 (2卷) 【西晉 竺法護譯】
  \item 禪祕要法經:《禪祕要法經》 (3卷) 【姚秦 鳩摩羅什等譯】
  \item 解深密經:《解深密經》 (5卷) 【唐 玄奘譯】
\end{itemize}
\subsubsection{丙}
\begin{itemize}
  \item 瑜伽師地論本地分(前五十卷):《瑜伽師地論》 (100卷) 【彌勒菩薩說 唐 玄奘譯】
  \item 新華嚴經論:《新華嚴經論》 (40卷) 【唐 李通玄撰】
  \item 入大乘論:《入大乘論》 (2卷) 【堅意菩薩造 北涼 道泰等譯】
  \item 寶性論(四卷):《究竟一乘寶性論》 (4卷) 【後魏 勒那摩提譯】
  \item 成實論:《成實論》 (16卷) 【訶梨跋摩造 姚秦 鳩摩羅什譯】
  \item 集異門足論:《阿毘達磨集異門足論》 (20卷) 【尊者舍利子說 唐 玄奘譯】
  \item 大乘起信論(參考:印順法師著,大乘起信論講記):《大乘起信論》 (1卷) 【馬鳴菩薩造 梁 真諦譯】
  \item 攝大乘論(參考:印順法師著,攝大乘論講記):《攝大乘論本》 (3卷) 【無著菩薩造 唐 玄奘譯】
  \item 中論(參考:印順法師著,中觀論頌講記):《中論》 (4卷) 【龍樹菩薩造 梵志青目釋 姚秦 鳩摩羅什譯】
\end{itemize}
\subsection{第二年}
一面仍然保持印度傳來的教典,一面放寬到中國祖師的著述
\subsubsection{甲}
\begin{itemize}
  \item 華嚴經(十住品、如來性起品):《大方廣佛華嚴經》(十住品、如來性起品) (80卷) 【唐 實叉難陀譯】
  \item 涅槃經獅子吼菩薩品:《大般涅槃經》 (36卷) 【宋 慧嚴等依泥洹經加之】
  \item 涅槃經獅子吼菩薩品:《大般涅槃經》 (40卷) 【北涼 曇無讖譯】
  \item 般若經第十六會(五九三卷起):《大般若波羅蜜多經(第401卷-第600卷)》 (200卷) 【唐 玄奘譯】
  \item 金光明最勝王經:《金光明最勝王經》 (10卷) 【唐 義淨譯】
  \item 維摩詰經肇註:《注維摩詰經》 (10卷) 【後秦 僧肇撰】
  \item 勝鬘夫人經:《勝鬘師子吼一乘大方便方廣經》 (1卷) 【劉宋 求那跋陀羅譯】
  \item 圓覺經:《大方廣圓覺修多羅了義經》 (1卷) 【唐 佛陀多羅譯】
  \item 分別緣起初勝法門經:《分別緣起初勝法門經》 (2卷) 【唐 玄奘譯】
  \item 大毘盧遮那成佛神變加持經:《大毘盧遮那成佛神變加持經》 (7卷) 【唐 善無畏.一行譯】
  \item 遺教經:《佛垂般涅槃略說教誡經》 (1卷) 【姚秦 鳩摩羅什譯】
\end{itemize}
\subsubsection{乙}
\begin{itemize}
  \item 瑜伽師地論攝事分(參考:印順法師著,《雜阿含經論會編》):《瑜伽師地論》 (100卷) 【彌勒菩薩說 唐 玄奘譯】
  \item 雜阿毘曇心論:《雜阿毘曇心論》 (11卷) 【尊者法救造 宋 僧伽跋摩等譯】
  \item 大智度論初品(共三十四卷):《大智度論》 (100卷) 【龍樹菩薩造 後秦 鳩摩羅什譯】
  \item 法華經論:《妙法蓮華經憂波提舍》 (2卷) 【大乘論師婆藪槃豆釋 後魏 菩提留支共曇林等譯】
  \item 大乘莊嚴經論:《大乘莊嚴經論》 (13卷) 【無著菩薩造 唐 波羅頗蜜多羅譯】
  \item 雜集論:《大乘阿毘達磨雜集論》 (16卷) 【安慧菩薩糅 唐 玄奘譯】
  \item 辨中邊論:《辯中邊論》 (3卷) 【世親菩薩造 唐 玄奘譯】
  \item 發菩提心經論:《發菩提心經論》 (2卷) 【天親菩薩造 後秦 鳩摩羅什譯】
  \item 發菩提心資糧論(六卷):《菩提資糧論》 (6卷) 【龍樹本 自在比丘釋 隋 達磨笈多譯】
  \item 佛母般若波羅蜜多圓集要義論:《佛母般若波羅蜜多圓集要義論》 (1卷) 【大域龍菩薩造 宋 施護等譯】
  \item 肇論:《肇論》 (1卷) 【後秦 僧肇作】
  \item 大乘義章:《大乘義章》 (20卷) 【隋 慧遠撰】
\end{itemize}
\subsubsection{丙}
\begin{itemize}
  \item 六祖壇經:《六祖大師法寶壇經》 (1卷) 【元 宗寶編】
  \item 法華遊意:《法華遊意》 (1卷) 【隋 吉藏造】
  \item 禪源諸詮集都序:《禪源諸詮集都序》 (2卷) 【唐 宗密述】
  \item 阿彌陀經要解:《阿彌陀經要解》 (1卷) 【明 智旭解】
  \item 觀無量壽經疏(善導著):《觀無量壽佛經疏》 (4卷) 【唐 善導集記】
  \item 淨土論注(曇鸞著):《略論安樂淨土義》 (1卷) 【後魏 曇鸞撰】
  \item 安樂集(道綽著):《安樂集》 (2卷) 【唐 道綽撰】
  \item 瑜伽菩薩戒釋(太虛大師全書.第八篇律釋):【網路資料】
  \item 天台菩薩戒疏(道綽著):《天台菩薩戒疏》 (3卷) 【唐 明曠刪補】
  \item 四分律比丘含註戒本(受比丘戒者):《四分律比丘含注戒本》 (3卷) 【唐 道宣述】
  \item 寶積經三律儀會(未受比丘戒者):《大寶積經》 (120卷) 【唐 菩提流志譯】
  \item 沙彌律儀增註(未受比丘戒者):《沙彌律儀要略增註》 (2卷) 【清 弘贊註】
  \item 維摩詰經玄疏:《維摩經玄疏》 (6卷) 【隋 智顗撰】
  \item 教觀綱宗:《教觀綱宗》 (1卷) 【明 智旭述】
  \item 天台傳佛心印記:《天台傳佛心印記》 (1卷) 【元 懷則述】
  \item 華嚴一乘教義分齊章:《華嚴一乘教義分齊章》 (4卷) 【唐 法藏述】
  \item 十二門論宗致義記:《十二門論宗致義記》 (2卷) 【唐 法藏述】
  \item 能顯中邊慧日論:《能顯中邊慧日論》 (4卷) 【唐 慧沼撰】
  \item 首楞嚴義疏注經:《首楞嚴義疏注經》 (10卷) 【宋 子璿集】
  \item 高僧傳(梁慧皎撰):《高僧傳》 (14卷) 【梁 慧皎撰】
\end{itemize}
\subsection{第三年}
擴展到暹羅、日本、西藏各家所傳作品。
\subsubsection{甲}
\begin{itemize}
  \item 華嚴經入法界品:《大方廣佛華嚴經》 (80卷) 【唐 實叉難陀譯】
  \item 父子合集經(二十卷):《父子合集經》 (20卷) 【宋 日稱等譯】
  \item 大寶積經(44)寶梁聚會(二卷):《大寶積經》 (120卷) 【唐 菩提流志譯】
  \item 寶星陀羅尼經(十卷):《寶星陀羅尼經》 (10卷) 【唐 波羅頗蜜多羅譯】
\end{itemize}
\subsubsection{乙}
\begin{itemize}
  \item 大智度論(第三十五卷起):《大智度論》 (100卷) 【龍樹菩薩造 後秦 鳩摩羅什譯】
  \item 十住毘婆沙論:《十住毘婆沙論》 (17卷) 【聖者龍樹造 後秦 鳩摩羅什譯】
  \item 成唯識論:《成唯識論》 (10卷) 【護法等菩薩造 唐 玄奘譯】
  \item 佛地經論:《佛地經論》 (7卷) 【親光菩薩等造 唐 玄奘譯】
\end{itemize}
\subsubsection{丙}
\begin{itemize}
  \item 圓覺經大疏釋義鈔(卍續第十四~十五冊):《圓覺經大疏釋義鈔》 (13卷) 【唐 宗密撰】
  \item 華嚴玄談(卍續第八冊):《華嚴經疏鈔玄談》 (9卷) 【唐 澄觀撰述】
  \item 法華玄義:《妙法蓮華經玄義》 (10卷) 【隋 智顗說】
  \item 釋禪波羅蜜次第法門:《釋禪波羅蜜次第法門》 (10卷) 【隋 智顗說】
  \item 宗鏡錄:《宗鏡錄》 (100卷) 【宋 延壽集】
  \item 黃蘗禪師語錄:《黃檗山斷際禪師傳心法要》 (1卷) 【唐 裴休集】
  \item 黃蘗禪師語錄:《黃檗斷際禪師宛陵錄》 (1卷) 【唐 裴休集】
\end{itemize}
\subsubsection{丁}
\begin{itemize}
  \item 續高僧傳:《續高僧傳》 (30卷) 【唐 道宣撰】
\end{itemize}

\section{南传}
\subsection{七论}
《法聚论》、《分别论》、《界论》、《人施设论》、《事论》、《双论》和《发趣论》
\paragraph{其它}
《清淨道論》

\section{藏传}
藏傳佛教是顯密圓融的體系,在顯宗上的弘傳過程中逐漸形成了以「五部大論」為核心的教法學制。
五部大論是指:因明學《釋量論》、般若學《現觀莊嚴論》、中觀學《入中論》、《俱舍論》、戒律學《戒律本論》。
按照黃教(格魯巴)的學習次第來說,大部份寺院裡首先學習五年因明學(含因類學及心類學一年),般若四年,中觀二年,《俱舍論》三年,律學四年等,需經十八年的學習,屬較長時間的學制。
\subsection{五部大论}
《現觀莊嚴論》
《入中論》
《釋量論》
《俱舍論》
《戒論》
\paragraph{其它}
《瑜伽师地论》



\section{藕益智旭大师}
《灵峰宗论》
\begin{itemize}
  \item 《法海观澜》
  \item 《阅藏知津》
\end{itemize}

\section{《瑜伽师地论》}
\subsubsection{《瑜伽師地論科句披尋記彙編》}
主要引用
\begin{itemize}
  \item 一經:《解深密經》
  \item 十二論:《顯揚聖教論》、《攝大乘論》、《世親攝論釋》、《成唯識論》、《俱舍論》、《集論》、《雜集論》、《集異門足論》、《成實論》、《辯中邊論》、《佛地經論》、《法蘊足論》
\end{itemize}
