\section{天台宗}

\subsection{传承}
慧文禪師
\footnote{北齊,讀大智度論三智實在一心中得文,及中論“因緣所生法 我說即是空 亦名為假名 亦名中道義”偈,頓悟龍樹即空即假即中之旨,立為心觀}
、南岳慧思
、智顗
\footnote{約略五時,開張八教,總括群籍,歸宗法華,說玄義以判教相,文句以解名義}
、章安灌頂
、法華智威
、天官慧威
、左溪玄朗
、繼體守文
、四明知禮
\footnote{山家乃四明之徒所自稱,以妄心為觀境,及許有事造三千者也}
\footnote{山外乃四明之徒所貶稱,以真如為觀境,及不許事造三千者也}

\subsection{教判}
\subsubsection{五时}
\begin{enumerate}
  \item 華嚴時
  \item 阿含時
  \item 方等時
  \item 般若時
  \item 法華涅槃時
\end{enumerate}

\subsubsection{八教}
化儀四教
\begin{enumerate}
  \item 頓教
  \item 漸教
  \item 秘密教
  \item 不定教
\end{enumerate}
化法四教
\begin{enumerate}
  \item 藏教
    \footnote{四阿含為經藏,毘尼為律藏,阿毘曇為論藏}
  \item 通教
  \item 別教
  \item 圓教
\end{enumerate}

\subsubsection{典籍}
此宗一家教門所用義旨,以妙法蓮華經為正依經典,以大智度論為指南,以涅槃經為扶疏以大品經為觀法,引維摩仁王等經以增信,引佛性寶性等論以助成。
一宗之學,以天台三大部
\footnote{法華玄義 法華文句 摩訶止觀}
為根本。他若南嶽大乘止觀,覺意三昧,天台釋禪波羅密次第法門,及六妙門等諸部止觀。
天台維摩玄義及疏,金光明玄義及疏,章安涅槃疏,荊溪釋籤疏記輔行,四明妙宗鈔等諸部疏鈔,以及荊溪之金錍義例,四明之十不二門指要鈔,幽溪之生無生論等著述,皆此宗之要藉,至諦觀之四教儀,蒲益之教觀綱宗,則此宗之入門也。

\subsubsection{教义}
圓融三諦
一念三千
三法無差
性具法門
無情有性
