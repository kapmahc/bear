\section{概要}

\begin{quote}
  「嚴淨毘尼,研究律藏」
  「毘尼久住,正法久住」\footnote{「六群比丘」}
\end{quote}

\begin{quote}
  「因時制宜,因地制宜」
  \footnote{大迦葉: 「佛所未制,今不別制,佛所已制,不可少改。」}
  \footnote{阿難: 「佛將入滅時曾告我,大眾若欲棄小小戒,可隨意棄。」}
\end{quote}

\begin{quote}
  「一切眾律中,戒經為最上;佛法三藏教,毘奈耶為首。」
\end{quote}

\begin{quote}
  「佛子離吾數千里,憶念吾戒,必證道果;在吾左右,雖常見吾,不順吾戒,終不得道。」
  \footnote{《四十二章經》}
\end{quote}


\subsection{大师}
\subsubsection{明朝紫柏大師} 脇不著席四十餘年,猶以未能持微細戒,終不敢為人授沙彌戒和比丘戒,到不得已時也只為人授五戒。
\subsubsection{明朝蕅益大師} 閱過律藏之後作「退戒緣起」,認為一向所受的戒不合法,沒有做比丘資格,自稱「菩薩沙彌」
\subsubsection{弘一律師} 嘗自檢驗,認為自己非但不夠比丘的資格,也不夠沙彌的資格,甚至連一個五戒滿分優婆塞的資格亦不夠。
\subsubsection{太虛大師} 「比丘不是佛未成,但願稱我為菩薩」
