\section{中国佛教史}
印度佛教最初傳來中國,是在釋尊入滅之後約五百多年,中國佛教傳到日本,則發生在第二個五百年之後。


\subsection{后漢}
\paragraph{明帝求法之說}
謂漢明帝夜夢求法見金人而知有佛陀之教,故派遣使節赴西域求取佛法。
\footnote{見於《後漢書.西域傳》等之記載。}
在途中遇到以白馬馱著經像的迦葉摩騰及竺法蘭兩位梵僧,於東漢明帝永平十年(西元六七年),歸至帝都洛陽門外建白馬寺,留居他們,並說由他們在那裡譯出《四十二章經》。
\footnote{《四十二章經》,乃是後世得自片斷經文的抄錄}
\paragraph{安世高}
在桓帝及靈帝(西元一六八~一八九年在位)時代來華,約二十年間,專心從事於經典的漢譯,譯出有《四諦經》、《轉法輪經》、《八正道經》、《安般守意經》等經典,達三十四部四十卷。
\footnote{這些都是小乘經典}
\paragraph{支婁迦讖(Lokaṣema)}
在桓帝之末,到達洛陽,於靈帝時代,譯出《道行般若經》、《般舟三昧經》、《首楞嚴經》、《無量清淨平等覺經》等,計十三部二十七卷。
\footnote{這些都屬於大乘經典。}

\subsection{魏晉}
在魏蜀吳三國鼎立的時代,活躍於江北的翻譯家,
有中印度的曇柯迦羅(Dharmakāla)\footnote{於魏廢帝嘉平二年(西元二五〇年)在洛陽譯出《僧祇戒本》}
、康居的康僧鎧(Saṃghavarman)、
安息的曇諦(Dharmottara)\footnote{曇諦譯出《四分律》的受戒作法《曇無德羯磨》}
等,在江南則有吳之支謙\footnote{一方面譯出《大阿彌陀經》、《維摩經》、《瑞應本起經》、《大般泥洹經》等,並且依據《無量壽經》及《中本起經》,製成《讚菩薩連句梵唄三契》,又為《了本生死經》作註解。}
、康居的康僧會\footnote{譯出闡說布施、持戒、忍辱、精進、禪定、智慧六波羅蜜的《六度集經》等}
等,值得注目。

\paragraph{朱士行} 依羯磨法而首先出家的中國人

\paragraph{竺法護}翻譯《光讚般若經》(二萬五千頌般若)、《正法華經》、《無量清淨平等覺經》等凡百五十四部三百零九卷
\footnote{有聶承遠及聶道真父子,參加譯場,因而開出了傳語、筆受和勸進等很多分工合作的譯經體制}
\subsubsection{漢人之出家}
在江南的漢族國家,始於東晉明帝太寧年間(西元三二三~三二五年),北地的胡族國家,始於後趙石虎的建武元年(西元三三五年)

\subsection{五代十国}
\subsubsection{道安}
\begin{enumerate}
  \item 經典目錄的作成
  \item 經典解釋
    \footnote{大旨多借用老莊的無為思想,闡明佛教的般若思想,此即所謂「格義佛教」,以竺法雅為首,康法朗和東晉竺潛的本無義、支遁(字道林)的即色義、竺法蘊的心無義等,追隨於沒。道安則起而排斥,注力於般若研究,以空來解釋一切諸法,本性空寂。}
    \footnote{後代將經典作為序文、正宗分、流通分三科分法的解釋之構成,傳說也是出於道安的發明,事實未必如此。}
  \item 僧制
    \footnote{從來僧人之姓,主要是以出生國名或師姓為準,漢人的僧名之上,一般多冠以安、支、康、帛、竺等之姓。道安與此相反,他以為出家者,悉奉釋尊之教,應以釋氏為姓,故自稱其名為釋道安。}
    \footnote{制定了僧尼軌範及佛法憲章三例,將一向雜然的中國僧團的行儀作法,作了計畫的統一。所謂三例者:1.行香、定座、上經、上講之法。2.平日的六時行道、飲食、唱時之法。3.布薩、差使、悔過等法。}
\end{enumerate}
\subsubsection{鳩摩羅什}
他的譯語,最為正確、流暢和適切,不落舊套,也使中國人最容易理解。
譯出經典達七十四部三百八十四卷,
\footnote{其中較受注目的可以數出《般若經》、《法華經》、《維摩經》、《彌陀經》等諸大乘經;《中論》、《百論》、《十二門論》、 大智度論》、《十住毘婆沙論》、《成實論》等。}
門下有弟子三千,達者八十,其中以僧肇\footnote{精通《維摩經》與《涅槃經》,所著之《註維摩經》《肇論》中的〈般若無知論〉}
、僧叡、道生
\footnote{著作《維摩經》、《法華經》、《泥洹經》等經的義疏之外,又有〈佛性當有論〉、〈法身無色論〉、〈佛無淨土論〉等的著述。}
\footnote{在當時江南佛教界風行著漸悟說的環境中,對他提倡頓悟成佛論的思想}
\footnote{基於法顯譯的《大般泥洹經》六卷,主張闡提成佛之義}
\footnote{後來由於曇無讖(Dharmarakṣa)所譯的四十卷《涅槃經》被傳到南方,佛教界始知他的主張正確而大為驚愕}
、道融、慧觀、道恆、僧導、曇影、慧叡、慧嚴等人為上選,特別以僧肇及道生二人最為傑出。
\subsubsection{覺賢(Buddhabhadra)佛陀跋陀羅}
原為什公的知交,因而來訪,然其主張不同,致與什公門下諍辯,便和慧觀等四十餘人,離開長安而赴廬山,初講禪經,又在建業之道場寺譯出《摩訶僧祇律》四十卷、《華嚴經》六十卷。
\footnote{他譯出的《華嚴經》,使得當時僅偏於般若教學的佛教界,激起了很大的漣漪。}
\subsubsection{曇無讖}
譯出《涅槃經》 四十卷(北本)
\footnote{由慧觀、慧嚴、謝靈運等,共同以法顯所譯《大般泥洹經》六卷,與此四十卷本作對校之後,重新編為三十六卷的《涅槃經》(南本)。}
\footnote{由此而對《涅槃經》的研究者盛行}
\subsubsection{廬山慧遠}
道安的高足弟子。
\footnote{與其俗弟慧持,同入道安之門,但在襄陽,由於兵亂,道安四散門人,因而別師,入江南廬山,住東林寺三十餘年,終其身未再下山,其間,送客亦以虎溪為界。}
\footnote{東晉安帝元興元年(西元四〇二年)與劉遺民及周續之等道俗名士二十三人,結社於東林寺般若臺,倡行念佛。世稱此一結社為白蓮社。}
\footnote{專以《般舟三昧經》為依據,念十方現在佛中之一的阿彌陀佛,此與後世彌陀念佛的立場略異,不過,此後中國的淨土教,則仰慧遠為淨土宗(蓮宗)的初祖。}
\footnote{遇到佛教思想的難解之處,或將門下送去羅什之處,或以書簡寄向什公質疑,那些書簡,現被收存於《大乘義章》中。}
\footnote{嘗作〈沙門不敬王者論〉,乃因當時的桓玄,質令沙門禮拜君王,故起而著論反對。}
\subsubsection{法顯}
慨於律藏之闕漏,於隆安三年(西元三九九年),和同學慧景、道整等,同由長安出發,遍參佛跡,途中同伴相繼離他而去,經錫蘭,義熙十年,由海路回國至青州(山東省)之時,僅他一人而已。隨後便在建業之道場寺,和覺賢共譯《摩訶僧祇律》四十卷,又譯《大般泥洹經》六卷,後來寂於荊州的幸寺,是年八十歲。
\footnote{他寫的旅行記《歷遊天竺紀傳》,被稱為《高僧法顯傳》,或《佛國記》,與唐玄奘《大唐西域記》,同為研究西域印度的永久指南。}


\subsection{南北朝}
\subsubsection{法社}
江南的佛教,因廬山慧遠的白蓮社,遂由貴族社會高蹈的思想議論,而發展為義學,此為形成法社的特色。法社的社友,要遵守社誡,也就是法社節度的制定,必須持戒和修道。
\footnote{從慧遠的〈法社節度序〉、〈外寺僧制度序〉、〈比丘尼節度序〉的撰作,可知當時除了白蓮社,尚有類似的法社存在。}
\subsubsection{義邑}
是由眾多的在家人為邑子,僧人為邑師,指導邑子而成佛教的團體
數十百位邑子,在化主邑師的勸導下,建造釋迦、 彌陀、彌勒、觀音等像,將此功德為求各自的父母、妻子、自己以及家族的現世利益和來世的願望,這種佛像的開光法會,稱為邑會。
\footnote{當時彌陀信仰者,可舉的有法曠、慧度、僧顯、慧宗、曇鑒、慧通等,也有願生兜率的傾向而發展成為彌勒信仰,例如道安及其門人為始,又有僧輔、智儼、道法等。又由於觀音信仰之利益的普遍,故有念觀音而使病癒之杯度、祈求航海安全之法純、念觀音而得妙音之帛法橋等}
\subsubsection{僧祇戶}
新成為北魏領土的山東地方平齊郡的郡戶,所應納於國庫的稅收,改納於僧曹,由僧曹管理,施捨給窮困者,以及維持官設的佛寺和造寺、法會等的事業費,特別是在饑饉災荒之際,用作賑濟。
\subsubsection{佛圖戶}
佛圖戶,是將犯了重刑的犯人,以及官之奴婢,移入佛寺管理,服清掃環境及寺田之耕作等雜役,同時接受佛教的感化教育。
\subsubsection{黑衣宰相慧琳}
作有《白黑論》(均善論),論儒佛之同異,以此為契機而引出何承天的《達性論》,慧遠門下宗炳之《難白黑論》、《明佛論》,顏延之(西元三四八~四五六年)的《釋何衡陽達性論》等連續對佛教教理作相互間的論難。
\subsubsection{译经}
\paragraph{南朝}
有於宋陽王景平元年(西元四二三年)來華譯出《五分律》的罽賓佛陀什(Buddhajīva),元嘉之初來華譯出《觀無量壽經》的西域人畺良耶舍(Kalayaśas 西元三八三-四三一年)。宋文帝元嘉八年(西元四三一年)至建康,僅九個月即以六十五歲圓寂而譯出了《菩薩善戒經》、《四分比丘尼羯磨法》等的求那跋摩,他使中國僧尼教團之受戒,成為可能。另有從海路自廣州登陸,於元嘉十二年至建康,受文帝優遇,後在荊州從事譯經,出有《雜阿含經》、《勝鬘經》、《過去現在因果經》等五十二部百四十四卷的求那跋陀羅(Guṇabhdra 西元三九四~四六八年)等。
\paragraph{齊}
譯出《無量義經》的曇摩伽陀耶舍(Dharmagatayaśas),譯出《善見律毘婆沙》的僧伽跋陀羅(Saṃgubhadra),譯出《百喻經》的求那毘地(Guṇavṛddhi),譯出《法華經.提婆達多品》的達摩摩提(Dharmamati)等人。
\subsubsection{僧传}
法雲的《法華義疏》,即為日本聖德太子撰述《法華義疏》的藍本。僧祐著有《出三藏記集》、《弘明集》、《釋迦譜》等史書,其弟子寶唱,亦受其影響,著《名僧傳》、《比丘尼傳》,慧皎便參考《名僧傳》
而撰述《高僧傳》。《比丘尼傳》及《高僧傳》兩書,與《出三藏記集》的僧傳,同為現存僧傳中最古而佔有極高評價的著作。
\subsubsection{梁武帝}
其信佛教之程度,乃為歷代帝王中所僅有絕無。
\footnote{天監三年(西元五〇四年)四月八日佛誕之期,率道俗二萬餘人,於重雲殿,行捨道奉佛儀式,同十年(西元五一一年),發表〈斷酒肉文〉,又於同十六年(西元五一七年),禁止以殺生做祭祀,並廢天下道觀,令道士還俗}
\footnote{同十八年(西元五一九年)四月八日,再從草堂寺之慧約受菩薩戒,當時自皇太子以下受戒者達四萬八千人。}
\footnote{大通二年(西元五二八年)三月,捨身同泰寺做三寶之奴僕,群臣出錢一億萬為武帝贖而歸,如此的捨身行助,此後又舉行了三次,故被呼為皇帝菩薩。}

\subparagraph{真諦}
翻譯,達四十九部百四十二卷,尤其譯出《攝大乘論》、《攝大乘論釋》、《大乘起信論》、《十七地論》、《決定藏論》、《中邊分別論》、《轉識論》、《金光明經》、《佛性論》、《唯識論》、《三無性論》、《阿毘達磨俱舍論釋論》,而使中國出現了攝論宗及俱舍宗,同時為唯識學開了研究的端緒,給佛教教學上作了一大開展。

\subsubsection{南嶽慧思}
著有《大乘止觀法門》、《無諍三昧法門》、《安樂行義》

\subsubsection{与道教}
道教,是起於漢末及三國時代的張角、張脩、張魯等所主倡的太平道及五斗米道,乃是誦《老子道德經》而利用符咒祈禱治病的民俗信仰,以此和同時代的左慈、葛玄等人倡導的神仙、養生、丹藥之方術,合流而成為組織化,產生天師、布行都講、祭酒、都督、主簿、姦令、鬼使等的職制,獲得了廣大的信徒,再加入老莊哲學的意味而成為道教。西晉時代,祭酒王浮,與河內的沙門帛遠(法祖),展開佛道之論爭,《老子化胡經》的著作,便是一個例證。這部《老子化胡經》,現已散佚,僅能見到其中少部分的內容,說什麼老子赴印度成為釋迦,或成為釋迦之師等說。佛教方面,也因此作了《天地經》、《清淨法行經》、《須彌四域經》、《空寂所問經》等,將中國的孔子稱為儒童菩薩, 老子呼為摩訶迦葉。
\\
東晉時代,有葛洪撰著《抱朴子》、《神仙傳》等道書。廬山道士陸修靜,奉宋明帝命,為建康的崇虛館主,以《上清真經》為始,增補道教經典,整理諸派道書,以洞真、洞玄、洞神之三洞分類,作成《玄覩觀經目》。又有梁道士陶弘景,隱居於茅山,作有《真話》七篇,將老子變為神格化,同時模仿佛典,偽作了很多的道經,遂成為上清派的創祖。其次,便是北魏寇謙之,使道教飛躍的發展成為國教。

\subsubsection{魏}
曇曜與吉迦夜(Kinkara)共同譯出《雜寶藏經》、《付法藏因緣傳》等。活躍於宣武帝治下的印度僧菩提流支、勒那摩提(Ratnamati)、佛陀扇多(Buddhaśānta)等三人,譯出有《華嚴經》之一品《十地經》、世親註解的《十地經論》,此與江南由真諦譯出的《攝大乘論》對峙,至此將尚未介紹給中國的世親思想引進來了。故有摩提門下的慧光(光統律師,南道派),流支門下的道寵(北道派)繼起,發展出了地論宗。分成南北二道派,又造成了重視《華嚴經》的風氣。

另外,流支譯有《金剛般若經》、《入楞伽經》、《無量壽經論》,摩提譯有《寶性論》,扇多譯有《攝大乘論》等。
在此期間,尚有一位主倡淨土信仰的曇鸞(西元四七六~五四二年),他是雁門(山西省)人,本習四論,因註解《大集經》感得氣疾,遂願求健康,訪梁道士陶弘景,得長生不老之法,北歸途中在洛陽遇到菩提流支,授之以《觀無量壽經》,從此,僅以佛法為長生不老之法,專心於念佛淨土之宣揚。
著有《淨土論註》、《讚阿彌陀佛偈》等,前者乃據流支傳譯的世親撰《淨土論》(《優婆提舍願生偈》)及龍樹撰《十住毘婆沙論.易行品》等說,而作的註解。將印度兩大主流思想,以淨土論的解釋,予以調和融會,使他得了新見解。他主張實踐禮拜、讚歎、作願、觀察、迴向之五念門,完成了中國淨土教的教理和實踐的基礎。

\subsection{佛教藝術}
佛教初傳中國,在西曆紀元前後之際,印度則正由部派執著而轉為大乘佛教運動之時,部派學者之中,也有不少與大乘教徒為伍而進出於西北印度。該地係由貴霜王朝(Ksāna)的外護而成為佛教的中心地帶。約在西元前三世紀以來,即有希臘人侵入而移住於該一地區,孔雀王朝以後,在大夏(Bactria)民族容受佛教的同時,便使希臘人曾運用萬神殿(Pantheion)的雕刻技術,創作佛像以作為禮拜的對象。這些佛像近世因在犍陀羅地域發掘出土,為數甚多,被稱為犍陀羅式佛像。但這卻是在印度廢棄了原來對於佛塔的崇拜之後的事,這從該地發現了好多塔基的遺跡和發掘出土的遺物可以推知。此種新佛像的雕刻法,傳入中印度佛教界,便形成了摩偷羅式,接著又產生了笈多式的佛像,並且大為盛行。

另一方面,犍陀羅的佛像雕刻,通過絲路(Silk Road),東傳中國而被容受,因此中國佛教建築物的佛像雕刻之初期作品,也是襲取這種來自西域印度的形式,但由於環境、風土、民族的不同,故亦隨著年代的進展,發達成為各自個別的形態。
\subsubsection{建筑}
印度的佛教建築,大別為塔(Stūpa)、支提(caitya塔、窟院)、毘訶羅(vihara僧院)。初以木造,後始用磚造或石造,或打成石窟。由於中國自古即在木造建築上發展,盛行所謂左右對稱式的宮殿樓閣的建築,中國佛教的建築,因此也多採用樓閣建築的架構法。
