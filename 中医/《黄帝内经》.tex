\chapter{《黄帝内经》}

\section{特点}
司外揣内\footnote{《管子$\cdot$地数篇》许多事物表里存在着相应的关系}、
援物比类\footnote{《素问$\cdot$示从容论》又名取象比类}、
直觉领悟\footnote{《素问$\cdot$八正神明论》非概念、非逻辑性的感性启示}、
揆度奇恒\footnote{《素问$\cdot$玉版论要》 用比较的方法测度事物的正常和异常}。

\begin{itemize}
  \item 天人合一,五脏一体\footnote{整体地把握生命规律}
  \item 人生有形,不离阴阳\footnote{辩证地对待生命活动}
  \item 侯之所始,道之所生\footnote{从功能概括生命本质}
\end{itemize}
