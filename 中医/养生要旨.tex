\chapter{养生要旨}

分为养形、养神两大类,基本原则是:形宜动,神应静,动静得宜,则“形与神俱,而尽终其天年”

\begin{quote}
  平人者不病也
\end{quote}
\begin{quote}
  形与神俱
\end{quote}
\begin{quote}
  顺四时而适寒暑,和喜怒而安居处
\end{quote}
\begin{quote}
  饮食有节,起居有常,不妄作劳。
\end{quote}
\begin{quote}
  虚邪贼风,避之有时。
\end{quote}

\section{神志}
\begin{quote}
  恬惔虚无\ 精神内守
\end{quote}

\section{饮食}
\begin{quote}
  食饮有节\ 谨和五味
\end{quote}

\section{劳作}
\begin{quote}
  形劳而不倦
\end{quote}

\section{禁忌}
\begin{quote}
  醉以人房,以欲竭其精,以耗散其真。
\end{quote}
