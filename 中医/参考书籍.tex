\chapter{参考书籍}

\section{教材}
\begin{itemize}
  \item 中医基础(李德新)
  \item 中医诊断学(朱文峰)
  \item 中药学(张廷模)
  \item 方剂学(邓中甲)
  \item 医古文(中国中医药出版社)
  \item 中国医学大成
\end{itemize}

\section{经典}
\begin{itemize}
  \item 《四圣心源》清代 黄元御
  \item 《黄帝内经》黄帝
  \item 《难经》扁鹊
  \item 《伤寒论》\footnote{张仲景所著《伤寒杂病论》,王叔和分为两部分:六经辨伤寒,脏腑论杂病}
  \item 《金匮要略》\footnote{同上}
  \item 《神农本草经》\footnote{上品药无毒,主益气;中品药或有毒或无毒,主治病、补虚;下品药有毒,主除病邪、破积聚。
  提出了“四气五味”的药性理论,“七情和合”的药物配伍理论}
  \item 《脉经》晋$\cdot$王叔和
  \item 《针灸甲乙经》晋$\cdot$皇甫谧
  \item 《诸病源侯论》隋$\cdot$巢元方
  \item 《千金要方》《千金翼方》唐$\cdot$孙思邈
\end{itemize}

\section{金元四大家}
\begin{itemize}
  \item 刘完素的火热说\ 寒凉派
  \item 张从正的攻邪说\ 攻邪派
  \item 李东垣的脾胃说\ 补土派
  \item 朱震亨的养阴说\ 滋阴派
\end{itemize}
