\section{精气学说}

\subsection{三才}
天、地、人

\subsection{气一元论}
萌生于先秦,成熟于战国及秦汉\footnote{西汉$\cdot$司马谈$\cdot$《论六家要旨》}。
认为气是构成万物的本原;
气既不是虚幻的,也不是超感觉的,是一种运行不息的物质,其存在的状态无非是弥散和聚合,即有形、无形两类;
有形和无形之间不仅没有不可逾越的鸿沟,而且随时处于相互转化之中;
万物的生成、变化、强盛、衰落都取决于气的运动及其变化。

\subsection{血、气}
血\footnote{属阴}与气\footnote{属阳}是维持人体生命活动的最基本物质\footnote{《素问$\cdot$调经论》《灵枢$\cdot$本藏》}。
人体的生、长、壮、老、衰,无不赖气以生存。
血由中焦脾胃受纳运化饮食水谷,吸收其中的精微物质,变化而成;
在脉中循行,内至五脏六腑,外达皮肉筋骨,起着濡养和滋润作用,保证生命活动的正常进行。

\subsection{人中之气来源}
\begin{itemize}
  \item 肾中精气
  \item 脾胃所化生的水谷之气
  \item 肺吸入之清气
\end{itemize}

\subsection{精}
禀受父母的生命物质与后天水谷精微相融合而成的精华物质\footnote{《素问$\cdot$金匮真言论》}。

\subsection{神}
七情\footnote{喜、怒、忧、思、悲、恐、惊}五神\footnote{神、魂、魄、意、志}的活动
