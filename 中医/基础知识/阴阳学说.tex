\section{阴阳学说}
作为一对既对立、又统一的范畴\footnote{相互对立、制约、排斥、互根、互用、互藏、交感、消长、转化、自和等运动规律和形式},
运用于诠解世界一切事物的相互关系及运动变化的规律\footnote{《易传$\cdot$系辞上》:一阴一阳之谓道}。

\begin{quote}
  阳化气,阴成形 \footnote{体内物质代谢的主要形式}
\end{quote}
\begin{quote}
  阴平阳秘\footnote{健康的象征}
\end{quote}
\begin{quote}
  察色按脉\ 先别阴阳
\end{quote}
\begin{quote}
  谨察阴阳所在而调之,以平为期\footnote{治疗的根本目的是协调阴阳}
\end{quote}

\begin{table}[H]
  \centering
  \caption[]{阴阳对照表}
  \begin{tabular}{|c|c|c|}
    \hline & 阳 & 阴 \\
    \hline 人身 & 背 & 腹 \\
    \hline 脏腑 & 腑 & 脏\\
    \hline
  \end{tabular}
\end{table}
