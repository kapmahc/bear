
\section{五行学说}
将自然界的许多事物或现象根据五行的属性特点,分为五大类别\footnote{《尚书$\cdot$洪范》};并认为它们之间存在着生克制化的作用。

\begin{table}[H]
  \centering
  \caption[]{五行对照表}
  \begin{tabular}{|c|c|c|c|c|c|}
    \hline & 金 & 木& 水 & 火 & 土 \\
    \hline 五脏 & & & & & \\
    \hline 六腑 & & & & & \\
    \hline 五气 & & & & & \\
    \hline 五神 & & & & & \\
    \hline 五志 & & & & & \\
    \hline 五体 & & & & & \\
    \hline 五时 & & & & & \\
    \hline 五味 & & & & & \\
    \hline 五色 & & & & & \\
    \hline 五音 & & & & & \\
    \hline 五声 & & & & & \\
    \hline
  \end{tabular}
\end{table}
