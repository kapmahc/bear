
\section{五行学说}
将自然界的许多事物或现象根据五行的属性特点,分为五大类别\footnote{《尚书$\cdot$洪范》};并认为它们之间存在着生克制化的作用。

\begin{table}[H]
  \centering
  \caption[]{五行对照表}
  \begin{tabular}{|c|c|c|c|c|c|}
    \hline & 火 & 木 & 土 & 金 & 水 \\
    \hline 季节 & 夏 & 春 & 长夏 & 秋 & 冬 \\
    \hline 脉象 & 弦 & 洪 & & 毛 & 石 \\
    \hline 五脏 & 心 & 肝 & 脾 & 肺 & 肾 \\
    \hline 六腑 & 小肠 & 胆 & 胃 & 大肠 & 膀胱 \\
    \hline 五气 & & & & & \\
    \hline 五神 & & & & & \\
    \hline 五志 & & & & & \\
    \hline 五体 & 脉 & 筋 & 肉 & 皮 & 骨 \\
    \hline 官窍 & 舌 & 目 & 口 & 鼻 & 耳及二阴 \\
    \hline 经脉(阳) & 手太阳小肠经 & 足少阳胆经 & 足阳明胃经 & 手阳明大肠经 & 足太阳膀胱经 \\
    \hline 经脉(阴) & 手少阴心经 & 足厥阴肝经 & 足太阴脾经 & 手太阴肺经 & 足少阴肾经 \\
    \hline 五时 & & & & & \\
    \hline 五味 & & & & & \\
    \hline 五色 & & & & & \\
    \hline 五音 & & & & & \\
    \hline 五声 & & & & & \\
    \hline
  \end{tabular}
\end{table}
