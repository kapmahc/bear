\section{防治}

\subsection{概念}
\paragraph{病} 即疾病,反映了某一疾病\textbf{全过程}的总体属性、特征及规律。
\paragraph{证} 即证侯,病机的外在反映,疾病过程中\textbf{某一阶段}或某一类型的病理概括;确定治法、处方遣药的依据。
\paragraph{症}
即症状和体征的总称,是疾病过程中表现出的个别、孤立的现象;
判断疾病,辨识\footnote{要求同时辨明疾病的病因、病位、病性、及其发展变化趋向}证侯的主要依据\footnote{但因其仅是疾病的个别现象,未必能完全反映疾病和证侯的本质,因而不能作为治疗的依据}。

\subsection{辨病因}

\subsection{辨病位}

\subsection{辨病性}

\subsection{辨病势}
\paragraph{传变规律}太阳->阳明->少阳->太阴->少阴->厥阴
\paragraph{温病学家}用卫气营血和上中下三焦来表示温热病和湿热病的传变规律
\paragraph{内伤杂病}《内经》用五行的生克乘侮规律来表达







\subsection{诊法}
\begin{quote}
  望、闻、问、切
\end{quote}
\begin{quote}
  因证立法、随法选方、据方施治。
\end{quote}

\subsection{原则}
\begin{quote}
  上工治未病
\end{quote}
\begin{quote}
  善治者,治皮毛
\end{quote}
\begin{quote}
  谨察阴阳所在而调之,以平为期
\end{quote}
\begin{quote}
  疏其血气,令其调达,而致和平。
\end{quote}
\begin{quote}
  上病下取、下病上取
\end{quote}
\begin{quote}
  从阴引阳,从阳引阴
\end{quote}
\begin{quote}
  其高者,因而越之;其下者,引而竭之。
\end{quote}
\begin{quote}
  治病必求其本
\end{quote}
\begin{quote}
  间者并行,甚者独行
\end{quote}
\begin{quote}
  圣人治病,必知天地阴阳,四时经纪
\end{quote}
\begin{quote}
  寒者热之,热者寒之\footnote{正治}
\end{quote}
\begin{quote}
  寒因寒用,热因热用\footnote{反治}
\end{quote}
\begin{quote}
  必先岁气,无伐天和
\end{quote}
