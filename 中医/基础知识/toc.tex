\chapter{基础知识}

\section{气一元论}
萌生于先秦,成熟于战国及秦汉\footnote{西汉$\cdot$司马谈$\cdot$《论六家要旨》}。
认为气是构成万物的本原;
气既不是虚幻的,也不是超感觉的,是一种运行不息的物质,其存在的状态无非是弥散和聚合,即有形、无形两类;
有形和无形之间不仅没有不可逾越的鸿沟,而且随时处于相互转化之中;
万物的生成、变化、强盛、衰落都取决于气的运动及其变化。

\section{阴阳学说}
作为一对既对立、又统一的范畴\footnote{相互对立、制约、排斥、互根、互用、互藏、交感、消长、转化、自和等运动规律和形式},
运用于诠解世界一切事物的相互关系及运动变化的规律\footnote{《易传$\cdot$系辞上》:一阴一阳之谓道}。

\begin{quote}
  阳化气,阴成形 \footnote{体内物质代谢的主要形式}
\end{quote}
\begin{quote}
  阴平阳秘\footnote{健康的象征}
\end{quote}
\begin{quote}
  察色按脉\ 先别阴阳
\end{quote}
\begin{quote}
  谨察阴阳所在而调之,以平为期\footnote{治疗的根本目的是协调阴阳}
\end{quote}

\begin{table}[H]
  \centering
  \caption[]{阴阳对照表}
  \begin{tabular}{|c|c|c|}
    \hline & 阳 & 阴 \\
    \hline 人身 & 背 & 腹 \\
    \hline 脏腑 & 腑 & 脏\\
    \hline
  \end{tabular}
\end{table}


\section{五行学说}
将自然界的许多事物或现象根据五行的属性特点,分为五大类别\footnote{《尚书$\cdot$洪范》};并认为它们之间存在着生克制化的作用。

\begin{table}[H]
  \centering
  \caption[]{五行对照表}
  \begin{tabular}{|c|c|c|c|c|c|}
    \hline & 金 & 木& 水 & 火 & 土 \\
    \hline 五脏 & & & & & \\
    \hline 六腑 & & & & & \\
    \hline 五气 & & & & & \\
    \hline 五神 & & & & & \\
    \hline 五志 & & & & & \\
    \hline 五体 & & & & & \\
    \hline 五时 & & & & & \\
    \hline 五味 & & & & & \\
    \hline 五色 & & & & & \\
    \hline 五音 & & & & & \\
    \hline 五声 & & & & & \\
    \hline
  \end{tabular}
\end{table}




\section{节气}
\begin{quote}
  五日谓之侯,三侯谓之气,六气谓之时,四时谓之岁。\footnote{《素问$cdot$六节藏象论》}
\end{quote}


\section{藏象}
以五脏为主体,将六腑、五体、五官、九窍、四肢百骸等全身器官分为五大系统。
之间并不孤立,通过经脉的络属沟通,气血的流贯,相互联系,形成统一的整体。

\section{血、气}
血\footnote{属阴}与气\footnote{属阳}是维持人体生命活动的最基本物质\footnote{《素问$\cdot$调经论》《灵枢$\cdot$本藏》}。
人体的生、长、壮、老、衰,无不赖气以生存。
血由中焦脾胃受纳运化饮食水谷,吸收其中的精微物质,变化而成;
在脉中循行,内至五脏六腑,外达皮肉筋骨,起着濡养和滋润作用,保证生命活动的正常进行。
\subsection{人中之气来源}
\begin{itemize}
  \item 肾中精气
  \item 脾胃所化生的水谷之气
  \item 肺吸入之清气
\end{itemize}

\section{精}
禀受父母的生命物质与后天水谷精微相融合而成的精华物质\footnote{《素问$\cdot$金匮真言论》}。

\section{神}
七情\footnote{喜、怒、忧、思、悲、恐、惊}五神\footnote{神、魂、魄、意、志}的活动

\section{三才}
天、地、人

\section{病机}
表里相传、循经传变、脏腑相移、循生克次第传变
\subsection{命名}
\begin{itemize}
  \item 病因
  \item 症状
  \item 病机
  \item 病位
\end{itemize}

\section{诊法}
望、闻、问、切





\section{讳饰}
\begin{table}[H]
  \centering
  \caption[]{讳饰对照表}
  \begin{tabular}{|c|c|c|}
    \hline 宗筋 & 男生殖器 \\
    \hline 阴器 & 女生殖器 \\
    \hline 茎 & 阴茎 \\
    \hline 垂 & 睾丸 \\
    \hline 子门 & 子宫之门 \\
    \hline 魄门 & 肛门 \\
    \hline
  \end{tabular}
\end{table}

\section{治疗}

\begin{quote}
  上工治未病
\end{quote}
\begin{quote}
  善治者,治皮毛
\end{quote}
\begin{quote}
  谨察阴阳所在而调之,以平为期
\end{quote}
\begin{quote}
  疏其血气,令其调达,而致和平。
\end{quote}
\begin{quote}
  上病下取、下病上取
\end{quote}
\begin{quote}
  从阴引阳,从阳引阴
\end{quote}
\begin{quote}
  其高者,因而越之;其下者,引而竭之。
\end{quote}
\begin{quote}
  治病必求其本
\end{quote}
\begin{quote}
  间者并行,甚者独行
\end{quote}
\begin{quote}
  圣人治病,必知天地阴阳,四时经纪
\end{quote}
\begin{quote}
  寒者热之,热者寒之\footnote{正治}
\end{quote}
\begin{quote}
  寒因寒用,热因热用\footnote{反治}
\end{quote}

\section{五运六气}

\paragraph{何谓五运}
木、火、土、金、水五行的运行规律。

\paragraph{何谓六气}
太阳、少阳、阳明、太阴、少阴、厥阴六经对应的风、寒、暑、湿、燥、火六种气化。

\subsection{节气}
\begin{quote}
  五日谓之侯,三侯谓之气,六气谓之时,四时谓之岁。\footnote{《素问$cdot$六节藏象论》}
\end{quote}

\section{经络系统}
人体通行气血、沟通表里上下、联络脏腑组织器官的一个系统\footnote{包括经脉、落脉、经别、经筋、皮部等}。
分为经脉\footnote{十二正经、奇经八脉、十二经别}、
络脉\footnote{别络、浮络、孙络}、内属脏腑部分、
外连体表部分\footnote{十二经筋、十二皮部}。

