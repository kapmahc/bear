\section{治疗}

\begin{quote}
  上工治未病
\end{quote}
\begin{quote}
  善治者,治皮毛
\end{quote}
\begin{quote}
  谨察阴阳所在而调之,以平为期
\end{quote}
\begin{quote}
  疏其血气,令其调达,而致和平。
\end{quote}
\begin{quote}
  上病下取、下病上取
\end{quote}
\begin{quote}
  从阴引阳,从阳引阴
\end{quote}
\begin{quote}
  其高者,因而越之;其下者,引而竭之。
\end{quote}
\begin{quote}
  治病必求其本
\end{quote}
\begin{quote}
  间者并行,甚者独行
\end{quote}
\begin{quote}
  圣人治病,必知天地阴阳,四时经纪
\end{quote}
\begin{quote}
  寒者热之,热者寒之\footnote{正治}
\end{quote}
\begin{quote}
  寒因寒用,热因热用\footnote{反治}
\end{quote}
