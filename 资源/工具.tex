\chapter{工具}

\paragraph{TEX Live}
本书使用\textbf{TEX Live}排版
\begin{enumerate}
  \item TEX Live 安装 \url{https://wiki.archlinux.org/index.php/TeX_Live_(%E7%AE%80%E4%BD%93%E4%B8%AD%E6%96%87)}
  \item TEX Live 指南 \url{https://www.tug.org/texlive/doc/texlive-zh-cn/texlive-zh-cn.pdf}
  \item \href{http://www.dralpha.com/zh/index.htm}{包老师}的\href{http://dralpha.altervista.org/zh/tech/lnotes2.pdf}{LaTeX笔记第二版}  
  \item DejaVu font\footnote{pacman -S ttf-dejavu ttf-arphic-ukai ttf-arphic-uming wqy-zenhei}
\end{enumerate}

\paragraph{dot}
本书图片使用\textbf{dot}生成
\begin{enumerate}
  \item 安装Graphviz \url{https://wiki.archlinux.org/index.php/Graphviz}
  \item 使用dot来绘图 \url{http://www.jianshu.com/p/5b02445eca1d}
\end{enumerate}

\paragraph{编辑器}
本书源文件使用\textbf{Atom}编辑
\begin{enumerate}
  \item The hackable text editor \url{https://github.com/atom/atom}
  \item Syntax highlighting and Snippets for LaTeX in atom. \url{https://atom.io/packages/language-latex}
\end{enumerate}

\paragraph{Unicode5.0}

\begin{enumerate}
  \item 标准CJK文字 \url{https://www.unicode.org/Public/5.0.0/ucd/Unihan.html}
  \item 全角ASCII、全角中英文标点、半宽片假名、半宽平假名、半宽韩文字母:FF00-FFEF \url{http://www.unicode.org/charts/PDF/UFF00.pdf}
  \item CJK部首补充:2E80-2EFF \url{http://www.unicode.org/charts/PDF/U2E80.pdf}
  \item CJK标点符号:3000-303F \url{http://www.unicode.org/charts/PDF/U3000.pdf}
  \item CJK笔划:31C0-31EF \url{http://www.unicode.org/charts/PDF/U31C0.pdf}
  \item 康熙部首:2F00-2FDF \url{http://www.unicode.org/charts/PDF/U2F00.pdf}
  \item 汉字结构描述字符:2FF0-2FFF \url{http://www.unicode.org/charts/PDF/U2FF0.pdf}
  \item 注音符号:3100-312F \url{http://www.unicode.org/charts/PDF/U3100.pdf}
  \item 注音符号(闽南语、客家语扩展):31A0-31BF \url{http://www.unicode.org/charts/PDF/U31A0.pdf}
  \item 日文平假名:3040-309F \url{http://www.unicode.org/charts/PDF/U3040.pdf}
  \item 日文片假名:30A0-30FF \url{http://www.unicode.org/charts/PDF/U30A0.pdf}
  \item 日文片假名拼音扩展:31F0-31FF \url{http://www.unicode.org/charts/PDF/U31F0.pdf}
  \item 韩文拼音:AC00-D7AF \url{http://www.unicode.org/charts/PDF/UAC00.pdf}
  \item 韩文字母:1100-11FF \url{http://www.unicode.org/charts/PDF/U1100.pdf}
  \item 韩文兼容字母:3130-318F \url{http://www.unicode.org/charts/PDF/U3130.pdf}
  \item 太玄经符号:1D300-1D35F \url{http://www.unicode.org/charts/PDF/U1D300.pdf}
  \item 易经六十四卦象:4DC0-4DFF \url{http://www.unicode.org/charts/PDF/U4DC0.pdf}
  \item 彝文音节:A000-A48F \url{http://www.unicode.org/charts/PDF/UA000.pdf}
  \item 彝文部首:A490-A4CF \url{http://www.unicode.org/charts/PDF/UA490.pdf}
  \item 盲文符号:2800-28FF \url{http://www.unicode.org/charts/PDF/U2800.pdf}
  \item CJK字母及月份:3200-32FF \url{http://www.unicode.org/charts/PDF/U3200.pdf}
  \item CJK特殊符号(日期合并):3300-33FF \url{http://www.unicode.org/charts/PDF/U3300.pdf}
  \item 装饰符号(非CJK专用):2700-27BF \url{http://www.unicode.org/charts/PDF/U2700.pdf}
  \item 杂项符号(非CJK专用):2600-26FF \url{http://www.unicode.org/charts/PDF/U2600.pdf}
  \item 中文竖排标点:FE10-FE1F \url{http://www.unicode.org/charts/PDF/UFE10.pdf}
  \item CJK兼容符号(竖排变体、下划线、顿号):FE30-FE4F \url{http://www.unicode.org/charts/PDF/UFE30.pdf}
\end{enumerate}
